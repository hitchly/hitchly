\documentclass[12pt, titlepage]{article}

\usepackage{booktabs}
\usepackage{tabularx}
\usepackage{float} % Add this for [H] table positioning
\usepackage{hyperref}
\hypersetup{
    colorlinks,
    citecolor=blue,
    filecolor=black,
    linkcolor=red,
    urlcolor=blue
}
\usepackage[round]{natbib}

%% Comments

\usepackage{color}

\newif\ifcomments\commentstrue %displays comments
%\newif\ifcomments\commentsfalse %so that comments do not display

\ifcomments
\newcommand{\authornote}[3]{\textcolor{#1}{[#3 ---#2]}}
\newcommand{\todo}[1]{\textcolor{red}{[TODO: #1]}}
\else
\newcommand{\authornote}[3]{}
\newcommand{\todo}[1]{}
\fi

\newcommand{\wss}[1]{\authornote{magenta}{SS}{#1}} 
\newcommand{\plt}[1]{\authornote{cyan}{TPLT}{#1}} %For explanation of the template
\newcommand{\an}[1]{\authornote{cyan}{Author}{#1}}

%% Common Parts

\newcommand{\progname}{ProgName} % PUT YOUR PROGRAM NAME HERE
\newcommand{\authname}{Team \#, Team Name
\\ Student 1 name
\\ Student 2 name
\\ Student 3 name
\\ Student 4 name} % AUTHOR NAMES                  

\usepackage{hyperref}
    \hypersetup{colorlinks=true, linkcolor=blue, citecolor=blue, filecolor=blue,
                urlcolor=blue, unicode=false}
    \urlstyle{same}
                                


\begin{document}

\title{System Verification and Validation Plan for \progname{}} 
\author{\authname}
\date{\today}
	
\maketitle

\pagenumbering{roman}

\section*{Revision History}

\begin{tabularx}{\textwidth}{p{3cm}p{2cm}X}
\toprule {\bf Date} & {\bf Version} & {\bf Notes}\\
\midrule
Date 1 & 1.0 & Notes\\
Date 2 & 1.1 & Notes\\
\bottomrule
\end{tabularx}

~\\
\wss{The intention of the VnV plan is to increase confidence in the software.
However, this does not mean listing every verification and validation technique
that has ever been devised.  The VnV plan should also be a \textbf{feasible}
plan. Execution of the plan should be possible with the time and team available.
If the full plan cannot be completed during the time available, it can either be
modified to ``fake it'', or a better solution is to add a section describing
what work has been completed and what work is still planned for the future.}

\wss{The VnV plan is typically started after the requirements stage, but before
the design stage.  This means that the sections related to unit testing cannot
initially be completed.  The sections will be filled in after the design stage
is complete.  the final version of the VnV plan should have all sections filled
in.}

\newpage

\tableofcontents

\listoftables
\wss{Remove this section if it isn't needed}

\listoffigures
\wss{Remove this section if it isn't needed}

\newpage

\section{Symbols, Abbreviations, and Acronyms}

\renewcommand{\arraystretch}{1.2}
\begin{tabular}{l l} 
  \toprule		
  \textbf{symbol} & \textbf{description}\\
  \midrule 
  T & Test\\
  FR & Functional Requirement\\
  NFR & Non-Functional Requirement\\
  V\&V & Verification and Validation\\
  SRS & Software Requirements Specification\\
  MG & Module Guide\\
  MIS & Module Interface Specification\\
  UI & User Interface\\
  DB & Database\\
  API & Application Programming Interface\\
  PoC & Proof of Concept\\
  QA & Quality Assurance\\
  CSV & Comma-Separated Values (data format)\\
  \bottomrule
\end{tabular}\\

\wss{symbols, abbreviations, or acronyms --- you can simply reference the SRS
  \citep{SRS} tables, if appropriate}

\wss{Remove this section if it isn't needed}

\newpage

\pagenumbering{arabic}

This document ... \wss{provide an introductory blurb and roadmap of the
  Verification and Validation plan}

\section{General Information}

\subsection{Summary}

\wss{Say what software is being tested.  Give its name and a brief overview of
  its general functions.}

\subsection{Objectives}

\wss{State what is intended to be accomplished.  The objective will be around
  the qualities that are most important for your project.  You might have
  something like: ``build confidence in the software correctness,''
  ``demonstrate adequate usability.'' etc.  You won't list all of the qualities,
  just those that are most important.}

\wss{You should also list the objectives that are out of scope.  You don't have 
the resources to do everything, so what will you be leaving out.  For instance, 
if you are not going to verify the quality of usability, state this.  It is also 
worthwhile to justify why the objectives are left out.}

\wss{The objectives are important because they highlight that you are aware of 
limitations in your resources for verification and validation.  You can't do everything, 
so what are you going to prioritize?  As an example, if your system depends on an 
external library, you can explicitly state that you will assume that external library 
has already been verified by its implementation team.}

\subsection{Challenge Level and Extras}

\wss{State the challenge level (advanced, general, basic) for your project.
Your challenge level should exactly match what is included in your problem
statement.  This should be the challenge level agreed on between you and the
course instructor.  You can use a pull request to update your challenge level
(in TeamComposition.csv or Repos.csv) if your plan changes as a result of the
VnV planning exercise.}

\wss{Summarize the extras (if any) that were tackled by this project.  Extras
can include usability testing, code walkthroughs, user documentation, formal
proof, GenderMag personas, Design Thinking, etc.  Extras should have already
been approved by the course instructor as included in your problem statement.
You can use a pull request to update your extras (in TeamComposition.csv or
Repos.csv) if your plan changes as a result of the VnV planning exercise.}

\subsection{Relevant Documentation}

\wss{Reference relevant documentation.  This will definitely include your SRS
  and your other project documents (design documents, like MG, MIS, etc).  You
  can include these even before they are written, since by the time the project
  is done, they will be written.  You can create BibTeX entries for your
  documents and within those entries include a hyperlink to the documents.}

\citet{SRS}

\wss{Don't just list the other documents.  You should explain why they are relevant and 
how they relate to your VnV efforts.}

\section{Plan}

\wss{Introduce this section.  You can provide a roadmap of the sections to
  come.}

\subsection{Verification and Validation Team}

\wss{Your teammates.  Maybe your supervisor.
  You should do more than list names.  You should say what each person's role is
  for the project's verification.  A table is a good way to summarize this information.}

\subsection{SRS Verification}

\wss{List any approaches you intend to use for SRS verification.  This may
  include ad hoc feedback from reviewers, like your classmates (like your
  primary reviewer), or you may plan for something more rigorous/systematic.}

\wss{If you have a supervisor for the project, you shouldn't just say they will
read over the SRS.  You should explain your structured approach to the review.
Will you have a meeting?  What will you present?  What questions will you ask?
Will you give them instructions for a task-based inspection?  Will you use your
issue tracker?}

\wss{Maybe create an SRS checklist?}

\subsection{Design Verification}

\wss{Plans for design verification}

\wss{The review will include reviews by your classmates}

\wss{Create a checklists?}

\subsection{Verification and Validation Plan Verification}

\wss{The verification and validation plan is an artifact that should also be
verified.  Techniques for this include review and mutation testing.}

\wss{The review will include reviews by your classmates}

\wss{Create a checklists?}

\subsection{Implementation Verification}

\wss{You should at least point to the tests listed in this document and the unit
  testing plan.}

\wss{In this section you would also give any details of any plans for static
  verification of the implementation.  Potential techniques include code
  walkthroughs, code inspection, static analyzers, etc.}

\wss{The final class presentation in CAS 741 could be used as a code
walkthrough.  There is also a possibility of using the final presentation (in
CAS741) for a partial usability survey.}

\subsection{Automated Testing and Verification Tools}

\wss{What tools are you using for automated testing.  Likely a unit testing
  framework and maybe a profiling tool, like ValGrind.  Other possible tools
  include a static analyzer, make, continuous integration tools, test coverage
  tools, etc.  Explain your plans for summarizing code coverage metrics.
  Linters are another important class of tools.  For the programming language
  you select, you should look at the available linters.  There may also be tools
  that verify that coding standards have been respected, like flake9 for
  Python.}

\wss{If you have already done this in the development plan, you can point to
that document.}

\wss{The details of this section will likely evolve as you get closer to the
  implementation.}

\subsection{Software Validation}

\wss{If there is any external data that can be used for validation, you should
  point to it here.  If there are no plans for validation, you should state that
  here.}

\wss{You might want to use review sessions with the stakeholder to check that
the requirements document captures the right requirements.  Maybe task based
inspection?}

\wss{For those capstone teams with an external supervisor, the Rev 0 demo should 
be used as an opportunity to validate the requirements.  You should plan on 
demonstrating your project to your supervisor shortly after the scheduled Rev 0 demo.  
The feedback from your supervisor will be very useful for improving your project.}

\wss{For teams without an external supervisor, user testing can serve the same purpose 
as a Rev 0 demo for the supervisor.}

\wss{This section might reference back to the SRS verification section.}

\section{System Tests}

Purpose: Verify that Hitchly meets its defined functional and non-functional requirements.

Structure: 
\begin{itemize}
    \item Section 4.1 tests core functional requirements (verification and matching).
    \item Section 4.2 tests key non-functional qualities (reliability and usability).
    \item Section 4.3 maps each test to its source requirement for traceability.
\end{itemize}

Reference: Section 1.4 (G.4) \textit{Functionality Overview} defines the functional and non-functional requirements.

\subsection{Tests for Functional Requirements}

Objective: Confirm correct operation of primary system functions—user verification and ride matching.

Justification: These address Hitchly’s two most critical risks trust and core value creation.

Sub-areas:
\begin{itemize}
    \item 4.1.1 User Verification Tests
    \item 4.1.2 Ride Matching Tests
\end{itemize}

\subsubsection{Area of Testing 1 – User Verification}

Purpose: Ensure only McMaster-affiliated users and verified drivers access the system.  
Covers: FR-1 (User verification) from Section 1.4 (G.4).

\begin{enumerate}
\item \textbf{Test Case 1 – Valid Sign-Up}\\
Control: Automatic\\
Initial State: No existing account.\\
Input: Valid McMaster email + password + driver license (upload for drivers).\\
Expected Output: Verification email sent → account activated after confirmation.\\
Test Case Derivation: Based on credential and email-domain constraints defined in Section 1.4 (G.4).\\
How test will be performed: Simulate sign-up, check database entry and email receipt.\\[5pt]

\item \textbf{Test Case 2 – Invalid Verification}\\
Control: Automatic\\
Initial State: No existing account.\\
Input: Non-McMaster email or invalid license format.\\
Expected Output: Account creation rejected with error message.\\
Test Case Derivation: Derived from input constraints and driver-license validation rules in Section 1.4 (G.4).\\
How test will be performed: Input invalid data, verify system blocks progress.\\
\end{enumerate}

\subsubsection{Area of Testing 2 – Ride Matching}

Purpose: Confirm that the matching algorithm correctly proposes compatible rider–driver pairs.  
Covers: FR-3 (Ride matching) and FR-4 (Cost estimate) from Section 1.4 (G.4).

\begin{enumerate}
\item \textbf{Test Case 1 – Successful Match}\\
Control: Automatic\\
Initial State: Verified users with valid profiles and schedules.\\
Input: Rider and driver with overlapping routes/times.\\
Expected Output: Match proposal generated with computed cost share.\\
Test Case Derivation: Based on matching criteria and cost-sharing logic specified in Section 1.4 (G.4).\\
How test will be performed: Run matching routine, check match records and cost calculation.\\[5pt]

\item \textbf{Test Case 2 – No Compatible Match}\\
Control: Automatic\\
Input: Rider and driver with non-overlapping schedules.\\
Expected Output: “No match found” message.\\
Test Case Derivation: From matching constraints outlined in Section 1.4 (G.4).\\
How test will be performed: Submit non-overlapping profiles and observe system response.\\
\end{enumerate}

\subsection{Tests for Nonfunctional Requirements}

Objective: Evaluate system quality beyond correctness—reliability and usability.  
Approach: Use performance measurement and user feedback methods rather than simple pass/fail criteria.

Sub-areas:
\begin{itemize}
    \item 4.2.1 Reliability (Matching Correctness)
    \item 4.2.2 Usability (Low-Friction Flows)
\end{itemize}

\subsubsection{Area of Testing 1 – Reliability (Matching Correctness)}

Purpose: Ensure consistent and repeatable ride matching results.  
Covers: NFR-1 (Reliability) from Section 1.4 (G.4).

\begin{enumerate}
\item \textbf{Test Case 1 – Repeatability Check}\\
Type: Automatic / Dynamic\\
Initial State: Stable dataset for verified users.\\
Input: Run matching inputs multiple times.\\
Expected Result: Identical pairings each run.\\
How test will be performed: Automated script logs output and compares hash results.\\[5pt]

\item \textbf{Test Case 2 – Stress Test}\\
Type: Dynamic Performance Test\\
Input: Simulated peak load ($\geq$ 1000 simultaneous match requests).\\
Expected Result: $\geq$ 95\% successful matches within response-time threshold.\\
How test will be performed: Load-testing tool records failures and latency for reliability assessment.\\
\end{enumerate}

\subsubsection{Area of Testing 2 – Usability (Low-Friction Flows)}

Purpose: Validate ease of use for key flows (sign-up, ride offer/request, confirmation).  
Covers: NFR-2 (Usability) from Section 1.4 (G.4).

\begin{enumerate}
\item \textbf{Test Case 1 – First-Time User Scenario}\\
Type: Manual / Survey-Based\\
Initial State: New participants with no training.\\
Input: Observe users completing sign-up and ride requests.\\
Expected Result: $\geq$ 90\% complete tasks without help in $\leq$ 3 minutes.\\
How test will be performed: Usability study plus survey metrics (Appendix).\\[5pt]

\item \textbf{Test Case 2 – Interface Clarity}\\
Type: Static Inspection / Heuristic Evaluation\\
Input: UI screens and labels.\\
Expected Result: No ambiguous labels or layout clutter violations.\\
How test will be performed: Walkthrough by UX reviewers using standard heuristics.\\
\end{enumerate}

\subsection{Traceability Between Test Cases and Requirements}

\begin{table}[H]
\centering
\begin{tabular}{|l|l|l|}
\hline
\textbf{Requirement ID} & \textbf{Requirement Description (from 1.4 G.4)} & \textbf{Test Case ID(s)} \\ \hline
FR-1 & User Verification & 4.1.1-1, 4.1.1-2 \\ \hline
FR-3 & Ride Matching & 4.1.2-1, 4.1.2-2 \\ \hline
FR-4 & Cost Estimate & 4.1.2-1 \\ \hline
NFR-1 & Reliability (Matching Correctness) & 4.2.1-1, 4.2.1-2 \\ \hline
NFR-2 & Usability (Low-Friction Flows) & 4.2.2-1, 4.2.2-2 \\ \hline
\end{tabular}
\caption{Traceability between test cases and requirements as defined in Section 1.4 (G.4) Functionality Overview.}
\end{table}



\section{Unit Test Description}

\wss{This section should not be filled in until after the MIS (detailed design
  document) has been completed.}

\wss{Reference your MIS (detailed design document) and explain your overall
philosophy for test case selection.}  

\wss{To save space and time, it may be an option to provide less detail in this section.  
For the unit tests you can potentially layout your testing strategy here.  That is, you 
can explain how tests will be selected for each module.  For instance, your test building 
approach could be test cases for each access program, including one test for normal behaviour 
and as many tests as needed for edge cases.  Rather than create the details of the input 
and output here, you could point to the unit testing code.  For this to work, you code 
needs to be well-documented, with meaningful names for all of the tests.}

\subsection{Unit Testing Scope}

\wss{What modules are outside of the scope.  If there are modules that are
  developed by someone else, then you would say here if you aren't planning on
  verifying them.  There may also be modules that are part of your software, but
  have a lower priority for verification than others.  If this is the case,
  explain your rationale for the ranking of module importance.}

\subsection{Tests for Functional Requirements}

\wss{Most of the verification will be through automated unit testing.  If
  appropriate specific modules can be verified by a non-testing based
  technique.  That can also be documented in this section.}

\subsubsection{Module 1}

\wss{Include a blurb here to explain why the subsections below cover the module.
  References to the MIS would be good.  You will want tests from a black box
  perspective and from a white box perspective.  Explain to the reader how the
  tests were selected.}

\begin{enumerate}

\item{test-id1\\}

Type: \wss{Functional, Dynamic, Manual, Automatic, Static etc. Most will
  be automatic}
					
Initial State: 
					
Input: 
					
Output: \wss{The expected result for the given inputs}

Test Case Derivation: \wss{Justify the expected value given in the Output field}

How test will be performed: 
					
\item{test-id2\\}

Type: \wss{Functional, Dynamic, Manual, Automatic, Static etc. Most will
  be automatic}
					
Initial State: 
					
Input: 
					
Output: \wss{The expected result for the given inputs}

Test Case Derivation: \wss{Justify the expected value given in the Output field}

How test will be performed: 

\item{...\\}
    
\end{enumerate}

\subsubsection{Module 2}

...

\subsection{Tests for Nonfunctional Requirements}

\wss{If there is a module that needs to be independently assessed for
  performance, those test cases can go here.  In some projects, planning for
  nonfunctional tests of units will not be that relevant.}

\wss{These tests may involve collecting performance data from previously
  mentioned functional tests.}

\subsubsection{Module ?}
		
\begin{enumerate}

\item{test-id1\\}

Type: \wss{Functional, Dynamic, Manual, Automatic, Static etc. Most will
  be automatic}
					
Initial State: 
					
Input/Condition: 
					
Output/Result: 
					
How test will be performed: 
					
\item{test-id2\\}

Type: Functional, Dynamic, Manual, Static etc.
					
Initial State: 
					
Input: 
					
Output: 
					
How test will be performed: 

\end{enumerate}

\subsubsection{Module ?}

...

\subsection{Traceability Between Test Cases and Modules}

\wss{Provide evidence that all of the modules have been considered.}
				
\bibliographystyle{plainnat}

\bibliography{../../refs/References}

\newpage

\section{Appendix}

This is where you can place additional information.

\subsection{Symbolic Parameters}

The definition of the test cases will call for SYMBOLIC\_CONSTANTS.
Their values are defined in this section for easy maintenance.

\subsection{Usability Survey Questions?}

\wss{This is a section that would be appropriate for some projects.}

\newpage{}
\section*{Appendix --- Reflection}

\wss{This section is not required for CAS 741}

The information in this section will be used to evaluate the team members on the
graduate attribute of Lifelong Learning.

The purpose of reflection questions is to give you a chance to assess your own
learning and that of your group as a whole, and to find ways to improve in the
future. Reflection is an important part of the learning process.  Reflection is
also an essential component of a successful software development process.  

Reflections are most interesting and useful when they're honest, even if the
stories they tell are imperfect. You will be marked based on your depth of
thought and analysis, and not based on the content of the reflections
themselves. Thus, for full marks we encourage you to answer openly and honestly
and to avoid simply writing ``what you think the evaluator wants to hear.''

Please answer the following questions.  Some questions can be answered on the
team level, but where appropriate, each team member should write their own
response:


\begin{enumerate}

\item \textbf{What went well while writing this deliverable?}\\
(Swesan) The structure of the V\&V plan came together smoothly once I reviewed the example templates and understood what was expected. I was able to clearly link each test to the functional and non-functional requirements from Section~1.4 (G.4), which made the writing process more organized. Formatting, tone consistency, and version control all went well, and I'm happy with how cohesive the final document feels.
\\
(Burhanuddin)
This document was very helpful in setting the basis for identifying and documenting the testing approaches for our capstone project. Specifically, the list of documentation was a very critical section that allowed me to understand the importance of each document and their connection to the specific tests. Overall, we had an in-depth discussion of the testing approaches we wanted to consider and documented them with full clarity.  

\item \textbf{What pain points did you experience during this deliverable, and how did you resolve them?}\\
(Swesan) The main difficulty was determining how detailed each test case should be and distinguishing between verification and validation tasks. At first, I wrote overly detailed descriptions, but after reviewing previous examples and clarifying expectations from the course materials, I refined the structure to focus on clarity and traceability. Once I aligned the test coverage with the proof-of-concept scope of Hitchly, the process became much more straightforward.
\\
(Burhanuddin)
Initially, the documentation had a few areas/sections that were very ambiguous in what they wanted. They lacked clarity in the description of the sections. Some of the descriptions were also outdated, which caused some confusion. Example, section 2 had a subsection for challenges which is part of an old rubric. Through the TA meetings, we were able to clarify what the description meant. 
\\
\item \textbf{What knowledge and skills will you need to acquire to successfully complete the verification and validation of your project?}\\
(Swesan) I need to deepen my understanding of dynamic testing and automation. Specifically, I want to improve at writing reproducible automated test scripts for functional verification and gain familiarity with load testing tools to assess performance under peak usage conditions. These skills will help ensure Hitchly's matching and verification features are both reliable and scalable.
\\
(Burhanuddin)
As I am fairly new to app development, I need to learn all types of web development testing frameworks. I also need to understand them in Typescript, a language we will be using for the development of the application.  


\item \textbf{For each of the knowledge areas and skills identified in the previous question, what are at least two approaches to acquiring the knowledge or mastering the skill? Which will you pursue, and why?}\\
(Swesan) To build my dynamic testing and automation skills, I can:  \\
(1) Review documentation and tutorials for testing frameworks and testing utilities.  \\
(2) Develop small-scale automated test cases directly within the Hitchly repository to gain hands-on experience.  \\

I will prioritize the second approach since applying testing concepts directly in code helps uncover integration issues early and strengthens confidence in both the testing process and implementation reliability.
\\
(Burhanuddin)
I will first look at language documentation and understand the basics of technology. Secondly, I will go through some video tutorials and get some hands-on practice with the desired testing tools.  


\end{enumerate}

\end{document}