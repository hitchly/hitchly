\documentclass[12pt, titlepage]{article}

\usepackage{booktabs}
\usepackage{tabularx}
\usepackage{float} % Add this for [H] table positioning
\usepackage{hyperref}
\hypersetup{
    colorlinks,
    citecolor=blue,
    filecolor=black,
    linkcolor=red,
    urlcolor=blue
}
\usepackage[round]{natbib}

%% Comments

\usepackage{color}

\newif\ifcomments\commentstrue %displays comments
%\newif\ifcomments\commentsfalse %so that comments do not display

\ifcomments
\newcommand{\authornote}[3]{\textcolor{#1}{[#3 ---#2]}}
\newcommand{\todo}[1]{\textcolor{red}{[TODO: #1]}}
\else
\newcommand{\authornote}[3]{}
\newcommand{\todo}[1]{}
\fi

\newcommand{\wss}[1]{\authornote{magenta}{SS}{#1}} 
\newcommand{\plt}[1]{\authornote{cyan}{TPLT}{#1}} %For explanation of the template
\newcommand{\an}[1]{\authornote{cyan}{Author}{#1}}

%% Common Parts

\newcommand{\progname}{ProgName} % PUT YOUR PROGRAM NAME HERE
\newcommand{\authname}{Team \#, Team Name
\\ Student 1 name
\\ Student 2 name
\\ Student 3 name
\\ Student 4 name} % AUTHOR NAMES                  

\usepackage{hyperref}
    \hypersetup{colorlinks=true, linkcolor=blue, citecolor=blue, filecolor=blue,
                urlcolor=blue, unicode=false}
    \urlstyle{same}
                                


\begin{document}

\title{System Verification and Validation Plan for \progname{}} 
\author{\authname}
\date{\today}
	
\maketitle

\pagenumbering{roman}

\section*{Revision History}

\begin{tabularx}{\textwidth}{p{3cm}p{2cm}X}
\toprule {\bf Date} & {\bf Version} & {\bf Notes}\\
\midrule
Date 1 & 1.0 & Notes\\
Date 2 & 1.1 & Notes\\
\bottomrule
\end{tabularx}

~\\
\wss{The intention of the VnV plan is to increase confidence in the software.
However, this does not mean listing every verification and validation technique
that has ever been devised.  The VnV plan should also be a \textbf{feasible}
plan. Execution of the plan should be possible with the time and team available.
If the full plan cannot be completed during the time available, it can either be
modified to ``fake it'', or a better solution is to add a section describing
what work has been completed and what work is still planned for the future.}

\wss{The VnV plan is typically started after the requirements stage, but before
the design stage.  This means that the sections related to unit testing cannot
initially be completed.  The sections will be filled in after the design stage
is complete.  the final version of the VnV plan should have all sections filled
in.}

\newpage

\tableofcontents

\listoftables
\wss{Remove this section if it isn't needed}

\listoffigures
\wss{Remove this section if it isn't needed}

\newpage

\section{Symbols, Abbreviations, and Acronyms}

\renewcommand{\arraystretch}{1.2}
\begin{tabular}{l l} 
  \toprule		
  \textbf{symbol} & \textbf{description}\\
  \midrule 
  T & Test\\
  FR & Functional Requirement\\
  NFR & Non-Functional Requirement\\
  V\&V & Verification and Validation\\
  SRS & Software Requirements Specification\\
  MG & Module Guide\\
  MIS & Module Interface Specification\\
  UI & User Interface\\
  DB & Database\\
  API & Application Programming Interface\\
  PoC & Proof of Concept\\
  QA & Quality Assurance\\
  CSV & Comma-Separated Values (data format)\\
  \bottomrule
\end{tabular}\\

\wss{symbols, abbreviations, or acronyms --- you can simply reference the SRS
  \citep{SRS} tables, if appropriate}

\wss{Remove this section if it isn't needed}

\newpage

\pagenumbering{arabic}

This document ... \wss{provide an introductory blurb and roadmap of the
  Verification and Validation plan}

\section{General Information}

\subsection{Summary}

\wss{Say what software is being tested.  Give its name and a brief overview of
  its general functions.}

\subsection{Objectives}

\wss{State what is intended to be accomplished.  The objective will be around
  the qualities that are most important for your project.  You might have
  something like: ``build confidence in the software correctness,''
  ``demonstrate adequate usability.'' etc.  You won't list all of the qualities,
  just those that are most important.}

\wss{You should also list the objectives that are out of scope.  You don't have 
the resources to do everything, so what will you be leaving out.  For instance, 
if you are not going to verify the quality of usability, state this.  It is also 
worthwhile to justify why the objectives are left out.}

\wss{The objectives are important because they highlight that you are aware of 
limitations in your resources for verification and validation.  You can't do everything, 
so what are you going to prioritize?  As an example, if your system depends on an 
external library, you can explicitly state that you will assume that external library 
has already been verified by its implementation team.}

\subsection{Challenge Level and Extras}

\wss{State the challenge level (advanced, general, basic) for your project.
Your challenge level should exactly match what is included in your problem
statement.  This should be the challenge level agreed on between you and the
course instructor.  You can use a pull request to update your challenge level
(in TeamComposition.csv or Repos.csv) if your plan changes as a result of the
VnV planning exercise.}

\wss{Summarize the extras (if any) that were tackled by this project.  Extras
can include usability testing, code walkthroughs, user documentation, formal
proof, GenderMag personas, Design Thinking, etc.  Extras should have already
been approved by the course instructor as included in your problem statement.
You can use a pull request to update your extras (in TeamComposition.csv or
Repos.csv) if your plan changes as a result of the VnV planning exercise.}

\subsection{Relevant Documentation}

\wss{Reference relevant documentation.  This will definitely include your SRS
  and your other project documents (design documents, like MG, MIS, etc).  You
  can include these even before they are written, since by the time the project
  is done, they will be written.  You can create BibTeX entries for your
  documents and within those entries include a hyperlink to the documents.}

\citet{SRS}

\wss{Don't just list the other documents.  You should explain why they are relevant and 
how they relate to your VnV efforts.}

\section{Plan}

This section discusses the team's plan for verification and validation of process implementation. It covers the teams and their roles, the SRS verification plan, the design verification plan, the verification and validation verification plan, the implementation verification plan, automated testing and verification tools, and the software validation plan. 

\subsection{Verification and Validation Team}

\section{Team Roles and Responsibilities}

\begin{tabular}{p{4cm} p{11cm}}

\textbf{Dr. Spencer Smith} \\ 
\textit{Advisor and Primary Documentation Reviewer} 
& Provides guidance for project direction, documentation structure, and technical feedback throughout the software development lifecycle. \\[1em]

\textbf{Lucas Dutton} \\ 
\textit{Advisor and Primary Documentation Reviewer} 
& Provides guidance for project direction, documentation structure, and technical feedback throughout the software development lifecycle. \\[1em]

\textbf{Sarim Zia} \\ 
\textit{Developer and Internal Lead Tester} 
& Responsible for reviewing team members’ pull requests to suggest improvements and/or fixes to code and documentation. 
Ensures smooth integration of reviewed features. Leads internal beta testing throughout the development phase and manages public beta testing to collect and review relevant feedback. \\[1em]

\textbf{Aidan Froggatt} \\ 
\textit{Developer and Code Quality Reviewer} 
& Reviews team members’ pull requests to suggest improvements and/or fixes to code and documentation. 
Oversees smooth integration of reviewed features through the creation of rulesets. 
Conducts internal beta testing throughout the development phase. \\[1em]

\textbf{Swesan Pathmanathan} \\ 
\textit{Developer and UI/UX Reviewer} 
& Reviews pull requests to identify code and documentation improvements. 
Ensures smooth integration of reviewed features. 
Conducts internal beta testing throughout the development phase, focusing on user interface responsiveness and overall usability. \\[1em]

\textbf{Hamzah Rawasia} \\ 
\textit{Developer and Functional Feature Tester} 
& Reviews pull requests to ensure adherence to coding standards and proper documentation. 
Conducts internal beta testing with an emphasis on validating functional features and performance reliability. \\[1em]

\textbf{Burhanuddin Kharodawala} \\ 
\textit{Developer and Non-Functional Feature Tester} 
& Reviews pull requests for quality and compliance with best practices. 
Ensures smooth integration of reviewed features. 
Conducts internal beta testing with a focus on non-functional requirements such as scalability, stability, and usability. \\

\end{tabular}

\subsection{SRS Verification}

We want to ensure that the SRS document is complete, accurate, and consistent, which will be verified using the following approaches: \\

\noindent \textbf{Peer Review:} The team will rely on feedback from classmates that are from other capstone teams for peer evaluation. It is expected that they will provide feedback on the clarity and completeness of each section, going through and finding any inconsistencies and flaws. Based on their findings, GitHub issues will be opened to provide specific suggestions for improvement and any aspects that can be improved. \\ 

\noindent \textbf{Stakeholder Feedback:} The primary stakeholder(s) of our project are McMaster University students and faculty that commute to campus. As such, we will share the SRS document with a sample of chosen stakeholders. This is important to ensure that the SRS reflects actual stakeholder needs and expectations rather than internal assumptions. The selected users will be able to give specific feedback on which features, requirements, and workflows meet their expectations of the functionality of the system. Any conflicts that may arise between user expectations and written requirements will trigger a revision of the corresponding sections. As for testing the requirements, users will be given an opportunity to use the system and  perform basic testing of functionality with set cases given to each user, where they will be able to see the workflow of the application first-hand and provide feedback on the functionality and whether it meets the requirements stated in the SRS. \\

\noindent \textbf{Team Validation:} The team will also meet internally to discuss the SRS and ensure the content is consistent with the intended functionlity and capabilities of the application, and is consistent throughout. Additionally, the team will also conduct our own testing using the system, especially for non-functional requirements to ensure that the application meets the performance requirements, as well as the overall functionality of each major component of the system. \\

\noindent \textbf{SRS Verification Checklist}
\begin{itemize}
    \item[{[ ]}] Are all of the core features (Matchmaking, scheduling, safety, payments, etc.) clearly described?
    \item[{[ ]}] Is every requirement unique and traceable?
    \item[{[ ]}] Has feedback from the classmate peer review been addressed?
    \item[{[ ]}] Are the requirements clearly defined and measurable?
    \item[{[ ]}] Are the inputs and outputs for important use cases clearly defined?
    \item[{[ ]}] Are there any conflicting requirements and/or assumptions?

\end{itemize}
\subsection{Design Verification}

To ensure the design is functional and up to standards, the team will conduct several reviews with all developers to thoroughly go over the technical soundness of the design and major artifacts, such as the system’s architecture, database schema, API specifications, and the UI/UX design. 
For the UI/UX design, it is especially important that we receive feedback from stakeholders to validate the user experience and flow. We will provide users with key user stories and the corresponding interface which is used to complete them so that they are able to get a feel of the functionality of the design. 
It is important that the team documents all feedback and any confusion from the users to validate the usability of the design. There will also be feedback received from peers on other capstone teams to verify design components, especially written components and diagrams. \\

\noindent \textbf{Design Verification Checklist:}
\begin{itemize}
  \item[{[ ]}] Does the database schema correctly model necessary fields for required data, and appropriate relationships? 
  \item[{[ ]}] Does the API design include secure, authenticated endpoints for core components and the functionality? 
  \item[{[ ]}] Does the system architecture clearly define how components will interact and how key features such as the matchmaking engine will work? 
  \item[{[ ]}] Do the UI wireframes clearly and intuitively display all core features and workflows? 
\end{itemize}
\subsection{Verification and Validation Plan Verification}

To ensure that the Verification and Validation plan is complete, correct, and effective, the team will hold an internal review meeting to ensure that the document is complete and feasible. We will also get peer feedback from classmates where they will identify areas of the plan which are unclear, incomplete, or needs improvement. 
To keep track of the feedback from peers, GitHub issues will be used so that peers can clearly communicate their thoughts. Additionally, the team will ensure that the tests have complete coverage of requirements, and will also make use of mutation testing to verify the quality and effectiveness of the unit tests in the plan. 
This involves introducing small faults into the code, such as changes in the matchmaking algorithm to modify efficiency and/or effectiveness and running the test suite against it to validate that it is robust. \\

\noindent \textbf{Verification and Validation Plan Verification:}
\begin{itemize}
  \item[{[ ]}] Do the requirements in the SRS have at least one corresponding test case?
  \item[{[ ]}] Are all test cases clear, specific, and testable, with defined expected outcomes? 
  \item[{[ ]}] Is the plan for mutation testing feasible?
  \item[{[ ]}] Has feedback from the peer review been addressed and closed? 
\end{itemize}
\subsection{Implementation Verification}

\begin{itemize}
  \item \textbf{Code Walkthrough/Inspection:} 
  New Code PR’s will be reviewed by at least two team members before the change is merged into the main branch. 
  The code will be reviewed via pull requests on GitHub and will include thoughtful titles and comments to ensure the reviewers are able to understand the changes. 
  If any further clarification is required, the developer will conduct a code walkthrough. 
  All feedback will be conveyed via GitHub, ensuring a standardized review procedure. 
  Only approved code will be merged into the main branch.

  \item \textbf{Unit Testing:} 
  Unit tests will be used to verify the expected behavior of the application. 
  Each module will have a corresponding unit test suite. 
  Unit test plans are outlined in Section 5 of this document.

  \item \textbf{System Testing:} 
  System testing will ensure all functional and non-functional requirements are met by the application. 
  Functional testing will verify key workflows, and non-functional testing will include performance, usability, and security checks. 
  System test plans are outlined in Section 4 of this document.

  \item \textbf{Static Analyzers:} 
  Static analysis tools will be used to analyze the code for formatting and good coding practices. 
  Reports generated from static analysis will be reviewed periodically to ensure high code quality. 
  Specific tools are listed under Section 3.6.
\end{itemize}

\subsection{Automated Testing and Verification Tools}

To ensure high quality, correct, and maintainable codebase, the team will use the following tools for automated testing and code verification, which were also briefly outlined in the Development Plan document. \\ 

\noindent \textbf{Unit Testing Framework:} Jest will be the primary framework for running unit tests for both the React Native frontend and Node.js backend logic. We will write unit tests for important business logic such as the cost-sharing calculation, parsing user timetables, and individual React components. \\
 
\noindent \textbf{Backend Integration Testing:} Supertest will be used to conduct integration testing on our backend API by simulating HTTP requests and verifying the responses. We will create test suites to verify that all API endpoints are functioning as expected, which includes testing for correct data responses and error handling. \\

\noindent \textbf{Static Analysis and Formatting:} ESLint and Prettier will ensure the codebase remains consistent, readable, and without common errors. ESLint will be configured to analyze our code for potential bugs and help enforce best practices, and Prettier will format code to maintain a consistent style across the codebase. \\

\noindent Jest will also be used for code coverage by utilizing the integrated coverage reporter to measure the percentage of our codebase that is executed by our unit and integration tests. Finally, we will also use Github Actions for Continuous Integration to automate the testing and verification process. \\

\subsection{Software Validation}

\section{Software Validation}

Our validation plan for the software includes extensive user testing and review sessions with all relevant stakeholders to ensure that all functional and non-functional requirements are met.

\begin{itemize}
    \item \textbf{User Testing:} 
    Hitchly relies on extensive external user testing to validate that the system functions correctly under real-world conditions. 
    Beta testing will begin early with a focus on core functionality. 
    Relevant feedback from beta users will be incorporated throughout the development phase. 
    Internal beta testing will also be performed to validate feature functionality and consistency before releasing it for public testing.

    \item \textbf{Stakeholder Review Sessions:} 
    Our stakeholder review sessions will be conducted regularly with our assigned TA to evaluate the system's progress against expected targets. 
    These meetings will be used to refine the system’s design and ensure stakeholder expectations are met.
\end{itemize}


\section{System Tests}

Purpose: Verify that Hitchly meets its defined functional and non-functional requirements.

Structure: 
\begin{itemize}
    \item Section 4.1 tests core functional requirements (verification and matching).
    \item Section 4.2 tests key non-functional qualities (reliability and usability).
    \item Section 4.3 maps each test to its source requirement for traceability.
\end{itemize}

Reference: Section 1.4 (G.4) \textit{Functionality Overview} defines the functional and non-functional requirements.

\subsection{Tests for Functional Requirements}

Objective: Confirm correct operation of primary system functions—user verification and ride matching.

Justification: These address Hitchly’s two most critical risks trust and core value creation.

Sub-areas:
\begin{itemize}
    \item 4.1.1 User Verification Tests
    \item 4.1.2 Ride Matching Tests
\end{itemize}

\subsubsection{Area of Testing 1 – User Verification}

Purpose: Ensure only McMaster-affiliated users and verified drivers access the system.  
Covers: FR-1 (User verification) from Section 1.4 (G.4).

\begin{enumerate}
\item \textbf{Test Case 1 – Valid Sign-Up}\\
Control: Automatic\\
Initial State: No existing account.\\
Input: Valid McMaster email + password + driver license (upload for drivers).\\
Expected Output: Verification email sent → account activated after confirmation.\\
Test Case Derivation: Based on credential and email-domain constraints defined in Section 1.4 (G.4).\\
How test will be performed: Simulate sign-up, check database entry and email receipt.\\[5pt]

\item \textbf{Test Case 2 – Invalid Verification}\\
Control: Automatic\\
Initial State: No existing account.\\
Input: Non-McMaster email or invalid license format.\\
Expected Output: Account creation rejected with error message.\\
Test Case Derivation: Derived from input constraints and driver-license validation rules in Section 1.4 (G.4).\\
How test will be performed: Input invalid data, verify system blocks progress.\\
\end{enumerate}

\subsubsection{Area of Testing 2 – Ride Matching}

Purpose: Confirm that the matching algorithm correctly proposes compatible rider–driver pairs.  
Covers: FR-3 (Ride matching) and FR-4 (Cost estimate) from Section 1.4 (G.4).

\begin{enumerate}
\item \textbf{Test Case 1 – Successful Match}\\
Control: Automatic\\
Initial State: Verified users with valid profiles and schedules.\\
Input: Rider and driver with overlapping routes/times.\\
Expected Output: Match proposal generated with computed cost share.\\
Test Case Derivation: Based on matching criteria and cost-sharing logic specified in Section 1.4 (G.4).\\
How test will be performed: Run matching routine, check match records and cost calculation.\\[5pt]

\item \textbf{Test Case 2 – No Compatible Match}\\
Control: Automatic\\
Input: Rider and driver with non-overlapping schedules.\\
Expected Output: “No match found” message.\\
Test Case Derivation: From matching constraints outlined in Section 1.4 (G.4).\\
How test will be performed: Submit non-overlapping profiles and observe system response.\\
\end{enumerate}

\subsection{Tests for Nonfunctional Requirements}

Objective: Evaluate system quality beyond correctness—reliability and usability.  
Approach: Use performance measurement and user feedback methods rather than simple pass/fail criteria.

Sub-areas:
\begin{itemize}
    \item 4.2.1 Reliability (Matching Correctness)
    \item 4.2.2 Usability (Low-Friction Flows)
\end{itemize}

\subsubsection{Area of Testing 1 – Reliability (Matching Correctness)}

Purpose: Ensure consistent and repeatable ride matching results.  
Covers: NFR-1 (Reliability) from Section 1.4 (G.4).

\begin{enumerate}
\item \textbf{Test Case 1 – Repeatability Check}\\
Type: Automatic / Dynamic\\
Initial State: Stable dataset for verified users.\\
Input: Run matching inputs multiple times.\\
Expected Result: Identical pairings each run.\\
How test will be performed: Automated script logs output and compares hash results.\\[5pt]

\item \textbf{Test Case 2 – Stress Test}\\
Type: Dynamic Performance Test\\
Input: Simulated peak load ($\geq$ 1000 simultaneous match requests).\\
Expected Result: $\geq$ 95\% successful matches within response-time threshold.\\
How test will be performed: Load-testing tool records failures and latency for reliability assessment.\\
\end{enumerate}

\subsubsection{Area of Testing 2 – Usability (Low-Friction Flows)}

Purpose: Validate ease of use for key flows (sign-up, ride offer/request, confirmation).  
Covers: NFR-2 (Usability) from Section 1.4 (G.4).

\begin{enumerate}
\item \textbf{Test Case 1 – First-Time User Scenario}\\
Type: Manual / Survey-Based\\
Initial State: New participants with no training.\\
Input: Observe users completing sign-up and ride requests.\\
Expected Result: $\geq$ 90\% complete tasks without help in $\leq$ 3 minutes.\\
How test will be performed: Usability study plus survey metrics (Appendix).\\[5pt]

\item \textbf{Test Case 2 – Interface Clarity}\\
Type: Static Inspection / Heuristic Evaluation\\
Input: UI screens and labels.\\
Expected Result: No ambiguous labels or layout clutter violations.\\
How test will be performed: Walkthrough by UX reviewers using standard heuristics.\\
\end{enumerate}

\subsection{Traceability Between Test Cases and Requirements}

\begin{table}[H]
\centering
\begin{tabular}{|l|l|l|}
\hline
\textbf{Requirement ID} & \textbf{Requirement Description (from 1.4 G.4)} & \textbf{Test Case ID(s)} \\ \hline
FR-1 & User Verification & 4.1.1-1, 4.1.1-2 \\ \hline
FR-3 & Ride Matching & 4.1.2-1, 4.1.2-2 \\ \hline
FR-4 & Cost Estimate & 4.1.2-1 \\ \hline
NFR-1 & Reliability (Matching Correctness) & 4.2.1-1, 4.2.1-2 \\ \hline
NFR-2 & Usability (Low-Friction Flows) & 4.2.2-1, 4.2.2-2 \\ \hline
\end{tabular}
\caption{Traceability between test cases and requirements as defined in Section 1.4 (G.4) Functionality Overview.}
\end{table}



\section{Unit Test Description}

\wss{This section should not be filled in until after the MIS (detailed design
  document) has been completed.}

\wss{Reference your MIS (detailed design document) and explain your overall
philosophy for test case selection.}  

\wss{To save space and time, it may be an option to provide less detail in this section.  
For the unit tests you can potentially layout your testing strategy here.  That is, you 
can explain how tests will be selected for each module.  For instance, your test building 
approach could be test cases for each access program, including one test for normal behaviour 
and as many tests as needed for edge cases.  Rather than create the details of the input 
and output here, you could point to the unit testing code.  For this to work, you code 
needs to be well-documented, with meaningful names for all of the tests.}

\subsection{Unit Testing Scope}

\wss{What modules are outside of the scope.  If there are modules that are
  developed by someone else, then you would say here if you aren't planning on
  verifying them.  There may also be modules that are part of your software, but
  have a lower priority for verification than others.  If this is the case,
  explain your rationale for the ranking of module importance.}

\subsection{Tests for Functional Requirements}

\wss{Most of the verification will be through automated unit testing.  If
  appropriate specific modules can be verified by a non-testing based
  technique.  That can also be documented in this section.}

\subsubsection{Module 1}

\wss{Include a blurb here to explain why the subsections below cover the module.
  References to the MIS would be good.  You will want tests from a black box
  perspective and from a white box perspective.  Explain to the reader how the
  tests were selected.}

\begin{enumerate}

\item{test-id1\\}

Type: \wss{Functional, Dynamic, Manual, Automatic, Static etc. Most will
  be automatic}
					
Initial State: 
					
Input: 
					
Output: \wss{The expected result for the given inputs}

Test Case Derivation: \wss{Justify the expected value given in the Output field}

How test will be performed: 
					
\item{test-id2\\}

Type: \wss{Functional, Dynamic, Manual, Automatic, Static etc. Most will
  be automatic}
					
Initial State: 
					
Input: 
					
Output: \wss{The expected result for the given inputs}

Test Case Derivation: \wss{Justify the expected value given in the Output field}

How test will be performed: 

\item{...\\}
    
\end{enumerate}

\subsubsection{Module 2}

...

\subsection{Tests for Nonfunctional Requirements}

\wss{If there is a module that needs to be independently assessed for
  performance, those test cases can go here.  In some projects, planning for
  nonfunctional tests of units will not be that relevant.}

\wss{These tests may involve collecting performance data from previously
  mentioned functional tests.}

\subsubsection{Module ?}
		
\begin{enumerate}

\item{test-id1\\}

Type: \wss{Functional, Dynamic, Manual, Automatic, Static etc. Most will
  be automatic}
					
Initial State: 
					
Input/Condition: 
					
Output/Result: 
					
How test will be performed: 
					
\item{test-id2\\}

Type: Functional, Dynamic, Manual, Static etc.
					
Initial State: 
					
Input: 
					
Output: 
					
How test will be performed: 

\end{enumerate}

\subsubsection{Module ?}

...

\subsection{Traceability Between Test Cases and Modules}

\wss{Provide evidence that all of the modules have been considered.}
				
\bibliographystyle{plainnat}

\bibliography{../../refs/References}

\newpage

\section{Appendix}

\subsection{Usability Testing Plan}
Usability testing will be conducted at the end of the first stage of development of the application. This will be used to assess and judge the user's compatibility with the application. This section will include a plan for conducting this test. 
\\
The goal is to conduct usability testing right after the development of the MVP of Hitchly. The MVP (minimal viable product) includes the ride matching algorithm, and user authentication. The plan is to find a group of students and staff to navigate through the application on their own and ask them for any feedback they may have. Secondly, ask the users a set of questions and observe their expressions; the time it takes them to complete a given task and features that may confuse them.  Below is a list of possible questions/tasks a user must complete to test the user compatibility of the application: 

\begin{itemize}
    \item Launch the application and register on the application.  
    \item Navigate to the user profile and add your timetable to the application. Update some information on your profile. 
    \item Navigate to the list of matched drivers/riders in the application.
    \item Select your profile type, Drivers or Riders.
    \item For Riders, click on the matched driver and access their user information.
    \item For Drivers, click on the matched rider and access their user information.
    \item Log out of the application and re-login to the application with the registered credentials in a given time constraint. 
\end{itemize}

Here are the main objectives of the test: 
\begin{itemize}
    \item Users find the process of creating an account intuitive and easy.
    \item Users are able to verify their emails quickly.
    \item Users are able to navigate through the various sections of the app without any problem.
    \item Users address any discomfort they have while using the application.  
    \item Users address any errors they face while using the application.
\end{itemize}

This brief test plan provides a structured approach for analyzing user's interaction with the application and understanding the key areas of improvement. Users will be monitored using a time metric to check if a user is able to complete a task within a given time constraint, ensuring efficiency. Moreover, users will be asked for feedback after each task they perform to get as much feedback as possible to improve the user experience.  

\newpage{}
\section*{Appendix --- Reflection}

\wss{This section is not required for CAS 741}

The information in this section will be used to evaluate the team members on the
graduate attribute of Lifelong Learning.

The purpose of reflection questions is to give you a chance to assess your own
learning and that of your group as a whole, and to find ways to improve in the
future. Reflection is an important part of the learning process.  Reflection is
also an essential component of a successful software development process.  

Reflections are most interesting and useful when they're honest, even if the
stories they tell are imperfect. You will be marked based on your depth of
thought and analysis, and not based on the content of the reflections
themselves. Thus, for full marks we encourage you to answer openly and honestly
and to avoid simply writing ``what you think the evaluator wants to hear.''

Please answer the following questions.  Some questions can be answered on the
team level, but where appropriate, each team member should write their own
response:


\begin{enumerate}

\item \textbf{What went well while writing this deliverable?}\\
(Swesan) The structure of the V\&V plan came together smoothly once I reviewed the example templates and understood what was expected. I was able to clearly link each test to the functional and non-functional requirements from Section~1.4 (G.4), which made the writing process more organized. Formatting, tone consistency, and version control all went well, and I’m happy with how cohesive the final document feels.

\item \textbf{What pain points did you experience during this deliverable, and how did you resolve them?}\\
(Swesan) The main difficulty was determining how detailed each test case should be and distinguishing between verification and validation tasks. At first, I wrote overly detailed descriptions, but after reviewing previous examples and clarifying expectations from the course materials, I refined the structure to focus on clarity and traceability. Once I aligned the test coverage with the proof-of-concept scope of Hitchly, the process became much more straightforward.

\item \textbf{What knowledge and skills will you need to acquire to successfully complete the verification and validation of your project?}\\
(Swesan) I need to deepen my understanding of dynamic testing and automation. Specifically, I want to improve at writing reproducible automated test scripts for functional verification and gain familiarity with load testing tools to assess performance under peak usage conditions. These skills will help ensure Hitchly’s matching and verification features are both reliable and scalable.

\item \textbf{For each of the knowledge areas and skills identified in the previous question, what are at least two approaches to acquiring the knowledge or mastering the skill? Which will you pursue, and why?}\\
(Swesan) To build my dynamic testing and automation skills, I can:  
(1) Review documentation and tutorials for testing frameworks and testing utilities.  
(2) Develop small-scale automated test cases directly within the Hitchly repository to gain hands-on experience.  

I will prioritize the second approach since applying testing concepts directly in code helps uncover integration issues early and strengthens confidence in both the testing process and implementation reliability.

\end{enumerate}

\end{document}