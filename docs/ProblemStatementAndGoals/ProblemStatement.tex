\documentclass{article}

\usepackage{tabularx}
\usepackage{booktabs}

\title{Problem Statement and Goals\\\progname}

\author{\authname}

\date{}

%% Comments

\usepackage{color}

\newif\ifcomments\commentstrue %displays comments
%\newif\ifcomments\commentsfalse %so that comments do not display

\ifcomments
\newcommand{\authornote}[3]{\textcolor{#1}{[#3 ---#2]}}
\newcommand{\todo}[1]{\textcolor{red}{[TODO: #1]}}
\else
\newcommand{\authornote}[3]{}
\newcommand{\todo}[1]{}
\fi

\newcommand{\wss}[1]{\authornote{magenta}{SS}{#1}} 
\newcommand{\plt}[1]{\authornote{cyan}{TPLT}{#1}} %For explanation of the template
\newcommand{\an}[1]{\authornote{cyan}{Author}{#1}}

%% Common Parts

\newcommand{\progname}{ProgName} % PUT YOUR PROGRAM NAME HERE
\newcommand{\authname}{Team \#, Team Name
\\ Student 1 name
\\ Student 2 name
\\ Student 3 name
\\ Student 4 name} % AUTHOR NAMES                  

\usepackage{hyperref}
    \hypersetup{colorlinks=true, linkcolor=blue, citecolor=blue, filecolor=blue,
                urlcolor=blue, unicode=false}
    \urlstyle{same}
                                


\begin{document}

\maketitle

\begin{table}[hp]
\caption{Revision History} \label{TblRevisionHistory}
\begin{tabularx}{\textwidth}{llX}
\toprule
\textbf{Date} & \textbf{Developer(s)} & \textbf{Change}\\
\midrule
Date1 & Name(s) & Description of changes\\
Date2 & Name(s) & Description of changes\\
... & ... & ...\\
\bottomrule
\end{tabularx}
\end{table}

\section{Problem Statement}

\wss{You should check your problem statement with the
\href{https://github.com/smiths/capTemplate/blob/main/docs/Checklists/ProbState-Checklist.pdf}
{problem statement checklist}.} 

\wss{You can change the section headings, as long as you include the required
information.}

\subsection{Problem}

\subsection{Inputs and Outputs}

\wss{Characterize the problem in terms of ``high level'' inputs and outputs.  
Use abstraction so that you can avoid details.}

\subsection{Stakeholders}

\subsection{Environment}

\wss{Hardware and Software Environment}

\section{Goals}

\section{Stretch Goals}

\section{Extras} 

To enhance the scope of the project, we would like to implement the following extra
activities:
\begin{itemize}
\item User Manual: This extra will allow for user-friendly documentation guiding them 
through our application.
\item Usability Testing: This extra will allow us to test the application in the environment
it was designed for, allowing us to observe the true usability/safety/sustainability of the application.
\end{itemize}


\newpage{}

\section*{Appendix --- Reflection}


The purpose of reflection questions is to give you a chance to assess your own
learning and that of your group as a whole, and to find ways to improve in the
future. Reflection is an important part of the learning process.  Reflection is
also an essential component of a successful software development process.  

Reflections are most interesting and useful when they're honest, even if the
stories they tell are imperfect. You will be marked based on your depth of
thought and analysis, and not based on the content of the reflections
themselves. Thus, for full marks we encourage you to answer openly and honestly
and to avoid simply writing ``what you think the evaluator wants to hear.''

Please answer the following questions.  Some questions can be answered on the
team level, but where appropriate, each team member should write their own
response:


\begin{enumerate}
    \item What went well while writing this deliverable? 
    Our team chose a project that was well suited for all of our background skills 
    through a brief period of discovery. We all were able to connect rather early before
    the group formation deadline, and this was a major factor in our success. Another important
    factor for our success was our ability to align our interests. In a group of five people, it
    can be difficult to pick a project that aligns with our interests, however we followed a voting 
    structure and ensured everyone had a chance to voice their opinions. The discovery phase for our 
    project allowed for a swift transition to our deliverable as we were able to finalize details early 
    on and focus more time on our research phase.
    \item What pain points did you experience during this deliverable, and how
    did you resolve them?
    Since our project was an idea that we came up with on our own, we did not have the benefit of having a supervisor 
    who can relay critical information to us during the research phase. However, this just meant we had the oppurtunity
    to dive deeper into the project through speaking to stakeholders around campus. Alongside that, we ensured to utilize our time 
    with the TA to ask as many questions as possible thus ensuring success in this deliverable.
    \item How did you and your team adjust the scope of your goals to ensure
    they are suitable for a Capstone project (not overly ambitious but also of
    appropriate complexity for a senior design project)?
    Adjusting the scope of our goals to ensure they were suitable for a Capstone project was supported by the help of our professor. During 
    the approval phase, we had a continous communication line with the professor to ensure we could incorporate his feedback. Through this, 
    and discussions within the team about technical skillset, we were able to come up with appropriate goals for the project.
    
\end{enumerate}  

\end{document}