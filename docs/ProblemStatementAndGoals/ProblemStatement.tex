\documentclass{article}
\usepackage[round]{natbib}
\usepackage{tabularx}
\usepackage[utf8]{inputenc}
\usepackage{booktabs}

\title{Problem Statement and Goals\\\progname}

\author{\authname}

\date{}

%% Comments

\usepackage{color}

\newif\ifcomments\commentstrue %displays comments
%\newif\ifcomments\commentsfalse %so that comments do not display

\ifcomments
\newcommand{\authornote}[3]{\textcolor{#1}{[#3 ---#2]}}
\newcommand{\todo}[1]{\textcolor{red}{[TODO: #1]}}
\else
\newcommand{\authornote}[3]{}
\newcommand{\todo}[1]{}
\fi

\newcommand{\wss}[1]{\authornote{magenta}{SS}{#1}} 
\newcommand{\plt}[1]{\authornote{cyan}{TPLT}{#1}} %For explanation of the template
\newcommand{\an}[1]{\authornote{cyan}{Author}{#1}}

%% Common Parts

\newcommand{\progname}{ProgName} % PUT YOUR PROGRAM NAME HERE
\newcommand{\authname}{Team \#, Team Name
\\ Student 1 name
\\ Student 2 name
\\ Student 3 name
\\ Student 4 name} % AUTHOR NAMES                  

\usepackage{hyperref}
    \hypersetup{colorlinks=true, linkcolor=blue, citecolor=blue, filecolor=blue,
                urlcolor=blue, unicode=false}
    \urlstyle{same}
                                


\begin{document}

\maketitle

\begin{table}[hp]
\caption{Revision History} \label{TblRevisionHistory}
\begin{tabularx}{\textwidth}{llX}
\toprule
\textbf{Date} & \textbf{Developer(s)} & \textbf{Change}\\
\midrule
September 21st, 2025 & Burhanuddin Kharodawala & Added Section 1 information\\
\bottomrule
\end{tabularx}
\end{table}

\section{Problem Statement}

\subsection{Problem}

The process of commuting to and from university presents numerous 
challenges. The major challenges include the 
increasing cost of fuel, GO bus fares, and the high 
cost of parking near the university premises. 
An average student and university staff/faculty 
member ends up paying more than 
\$150/month to \$250/month just on transportation ~\citep{ChandaAndSo2025}.
These not only put a financial strain on them but also 
add an emotional burden. This money could be well spent somewhere else. 
An example could be their home loans or their housing rent. 
They usually have to be very conservative on their leisure spending to save for these high transportation costs.
Moreover, the impact 
of commuters commuting to universities with their 
personal vehicles leave a significant impact on 
the environment. Universities Canada`s 2024 
footprint report states that \textbf{single occupancy} vehicle commuting contributed to around 
36\% of GHG emissions ~\citep{Kebirungi2024}. 
The air quality significantly drops due to the pollusion caused by these emmisions, which negatively affects people`s health.
To tackle these problems, commuters choose 
to carpool. However, even these bring new 
challenges. Applications like Kijiji, 
Facebook Marketplace, Poparide provide a 
solution to this problem; however, they come 
with minimal safety measures. These applications 
have drivers coming from all sorts of backgrounds 
with minimal verifications. They lack a centralized 
verification system to track driver`s details and background. 
\textbf{Hitchly} is an application that aims to make the process of commuting to McMaster 
University and back home, streamlined, affordable and safe for its users, and 
sustainable for the environment. 

\subsection{Inputs and Outputs}

\subsubsection{Inputs}
\begin{enumerate}
\item \textbf{User data:} The user adds their personal information, commute details, university email, timetable and their location.
\item \textbf{Driver Data:} The drivers specifically add their vehicle and license details in the app. They also add their cost estimation of fuel and parking into the app.
\end{enumerate}

\subsubsection{Outputs}
\begin{enumerate}
\item \textbf{User data:} Hitchly will match users (drivers and riders) based on their schedule and location. The app will also output a summary of the trip upon completion.
\item \textbf{Driver Data:} Hitchly will generate a price estimate for the driver to decide the price per rider for a trip.  
\end{enumerate}

\subsection{Stakeholders}

\subsubsection{Direct Stakeholders}
The main direct stakeholders are \textbf{McMaster students} and \textbf{McMaster staff} and \textbf{faculty members}. They commute to the university and home frequently for studies and work and often look for opportunities to save and earn extra money. They can use this application to either provide a rideshare service as drivers or use those services as riders.  

\subsubsection{Indirect Stakeholders}
\begin{enumerate}
\item \textbf{McMaster University:} The university will see a reduction in the congestion of traffic near its premises because of the application`s carpooling services.
\item \textbf{Hamilton Community and McMaster University Neighborhood:} The area will see a lot less congestion due to the reduction of cars. Moreover, it will also see a significant reduction in GHG emissions.  
\item \textbf{Student and Staff Families:} They will have a sense of relief and peace of mind for their carpooling family members due to the increased reliability and credibility of the ride sharing app. 
\end{enumerate}

\subsection{Environment}
This application will be compatible on all mobile devices. It will also be supported by the two major operating systems, Android and IOS. GitHub will be used for version controlling and VS Code will be used for coding and developing the application.

\section{Goals}

\section{Stretch Goals}

\section{Extras}

\wss{For CAS 741: State whether the project is a research project. This
designation, with the approval (or request) of the instructor, can be modified
over the course of the term.}

\wss{For SE Capstone: List your extras.  Potential extras include usability
testing, code walkthroughs, user documentation, formal proof, GenderMag
personas, Design Thinking, etc.  (The full list is on the course outline and in
Lecture 02.) Normally the number of extras will be two.  Approval of the extras
will be part of the discussion with the instructor for approving the project.
The extras, with the approval (or request) of the instructor, can be modified
over the course of the term.}

\newpage{}

\section*{Appendix --- Reflection}

\wss{Not required for CAS 741}

The purpose of reflection questions is to give you a chance to assess your own
learning and that of your group as a whole, and to find ways to improve in the
future. Reflection is an important part of the learning process.  Reflection is
also an essential component of a successful software development process.  

Reflections are most interesting and useful when they're honest, even if the
stories they tell are imperfect. You will be marked based on your depth of
thought and analysis, and not based on the content of the reflections
themselves. Thus, for full marks we encourage you to answer openly and honestly
and to avoid simply writing ``what you think the evaluator wants to hear.''

Please answer the following questions.  Some questions can be answered on the
team level, but where appropriate, each team member should write their own
response:


\begin{enumerate}
    \item What went well while writing this deliverable? 
    \item What pain points did you experience during this deliverable, and how
    did you resolve them?
    \item How did you and your team adjust the scope of your goals to ensure
    they are suitable for a Capstone project (not overly ambitious but also of
    appropriate complexity for a senior design project)?
\end{enumerate}  

\bibliographystyle {plainnat}
\bibliography{../../refs/References}


\end{document}


