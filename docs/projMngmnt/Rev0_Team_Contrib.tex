\documentclass{article}

\usepackage{float}
\restylefloat{table}

\usepackage{booktabs}

\title{Team Contributions: Rev 0\\\progname}

\author{\authname}

\date{}

\input{../Comments}
%% Common Parts

\newcommand{\progname}{Software Engineering} % PUT YOUR PROGRAM NAME HERE
\newcommand{\authname}{Team \#16, The Chill Guys
\\ Hamzah Rawasia
\\ Sarim Zia
\\ Aidan Froggatt
\\ Swesan Pathmanathan
\\ Burhanuddin Kharodawala} % AUTHOR NAMES                  

\usepackage{hyperref}
    \hypersetup{colorlinks=true, linkcolor=blue, citecolor=blue, filecolor=blue,
                urlcolor=blue, unicode=false}
    \urlstyle{same}
                                


\begin{document}

\maketitle

\section{Demo Plans}
For this demonstration, our team plans to showcase the core functionalities of our application.
This will build on top of the POC and this time, it will showcase the application's functionality with a clear \textbf{User Interface}. 
This functionalities include user authentication, data processing, matchmaking algorithm and other core features that set the basis of the rideshare application.
The demonstartion will involve a live walkthrough of the application functionalities. For instance, we will run a quick scenario, in which a user
signs-up/logs-in, selects their user type (driver or rider), and show their respective functionalities like matchmaking, trip management, reporting and cost estimation. Finally, the demostration will also include a 
breif overview of the admin dashboard which aims to handle user management, sfety reports and user/trip analytics. 
Overall, this demonstration will highlight the progress that has been made since the POC and demonstarte a user-friendly interface that is ready to enter its user testing phase.

\section{Team Meeting Attendance}

\wss{For each team member how many team meetings have they attended over the
time period of interest.  This number should be determined from the meeting
issues in the team's repo.  The first entry in the table should be the total
number of team meetings held by the team.}

\begin{table}[H]
\centering
\begin{tabular}{ll}
\toprule
\textbf{Student} & \textbf{Meetings}\\
\midrule
Total & 5\\
Hamzah Rawasia & 5\\
Sarim Zia & 5\\
Aidan Froggatt & 5\\
Swesan Pathmanathan & 5\\
Burhanuddin Kharodawala & 5\\
\bottomrule
\end{tabular}
\end{table}

\wss{If needed, an explanation for the counts can be provided here.}

\section{Supervisor/Stakeholder Meeting Attendance}

\noindent \textbf{McMaster Students} [fill in this information]

\begin{table}[H]
\centering
\begin{tabular}{ll}
\toprule
\textbf{Student} & \textbf{Meetings}\\
\midrule
Total & 2\\
Hamzah Rawasia & 2\\
Sarim Zia & 2\\
Aidan Froggatt & 2\\
Swesan Pathmanathan & 2\\
Burhanuddin Kharodawala & 2\\
\bottomrule
\end{tabular}
\end{table}

\section{Lecture Attendance}

\begin{table}[H]
\centering
\begin{tabular}{ll}
\toprule
\textbf{Student} & \textbf{Lectures}\\
\midrule
Total & 1\\
Hamzah Rawasia & 1\\
Sarim Zia & 1\\
Aidan Froggatt & 1\\
Swesan Pathmanathan & 1\\
Burhanuddin Kharodawala & 1\\
\bottomrule
\end{tabular}
\end{table}

\section{TA Document Discussion Attendance}

\noindent \textbf{TA's Name: Lucas Dutton}

\begin{table}[H]
\centering
\begin{tabular}{ll}
\toprule
\textbf{Student} & \textbf{Lectures}\\
\midrule
Total & 1\\
Hamzah Rawasia & 1\\
Sarim Zia & 1\\
Aidan Froggatt & 1\\
Swesan Pathmanathan & 1\\
Burhanuddin Kharodawala & 0\\
\bottomrule
\end{tabular}
\end{table}

Burhanuddin Kharodawala was unable to attend the TA document discussion as he was out of town that day.

\section{Commits}

\wss{For each team member how many commits to the main branch have been made
over the time period of interest.  The total is the total number of commits for
the entire team since the beginning of the term.  The percentage is the
percentage of the total commits made by each team member.}

\begin{table}[H]
\centering
\begin{tabular}{lll}
\toprule
\textbf{Student} & \textbf{Commits} & \textbf{Percent}\\
\midrule
Total & Num & 100\% \\
Name 1 & Num & \% \\
Name 2 & Num & \% \\
Name 3 & Num & \% \\
Name 4 & Num & \% \\
Name 5 & Num & \% \\
\bottomrule
\end{tabular}
\end{table}

\wss{If needed, an explanation for the counts can be provided here.  For
instance, if a team member has more commits to unmerged branches, these numbers
can be provided here.  If multiple people contribute to a commit, git allows for
multi-author commits.}

\section{Issue Tracker}

\wss{For each team member how many issues have they authored (including open and
closed issues (O+C)) and how many have they been assigned (only counting closed
issues (C only)) over the time period of interest.}

\begin{table}[H]
\centering
\begin{tabular}{lll}
\toprule
\textbf{Student} & \textbf{Authored (O+C)} & \textbf{Assigned (C only)}\\
\midrule
Name 1 & Num & Num \\
Name 2 & Num & Num \\
Name 3 & Num & Num \\
Name 4 & Num & Num \\
Name 5 & Num & Num \\
\bottomrule
\end{tabular}
\end{table}

\wss{If needed, an explanation for the counts can be provided here.}

\section{CICD}

\wss{Say how CICD is used in your project}

\section{Team Charter Trigger Items}

\wss{Provide a summary of the quantified triggers identified in the team's
charter.}

\wss{Provide a list of any violations of the triggers.  If the team wishes, the
violations can be summarized on aggregate, instead of naming specific team
members.}

\wss{Provide a plan to address the violations.  This could include revising the
triggers, if they are found to be too weak, strong or ambiguous.}

\section{Additional Productivity Metrics}

\wss{If your team has additional metrics of productivity, please feel free to
add them to this report.}

\end{document}