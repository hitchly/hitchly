% Sources:
% - .github/workflows/ci.yml (lines 1-264): Main CI workflow with code and LaTeX checks
% - .github/workflows/buildtex.yml (lines 1-58): Automated PDF generation workflow
% - docs/DevelopmentPlan/DevelopmentPlan.tex (lines 114-115): CI/CD overview and requirements

\section{CICD}

Hitchly employs a comprehensive Continuous Integration and Continuous Deployment (CI/CD) pipeline using GitHub Actions to ensure consistent quality, reliability, and maintainability across all project components. The pipeline is designed with intelligent path-based filtering to optimize build times by only running relevant checks when specific file types are modified.

\subsection{Continuous Integration}

The CI pipeline consists of two main workflows: \texttt{ci.yml} and \texttt{buildtex.yml}. The \texttt{ci.yml} workflow implements a change detection system that filters jobs based on whether code files or LaTeX documentation files were modified, ensuring efficient resource usage.

\subsubsection{Code Quality Checks}

% Source: .github/workflows/ci.yml (lines 39-110): Code quality check jobs (type-check, lint, format-check, build, test)

For code changes (TypeScript, JavaScript, JSON, Markdown, and configuration files), the following automated checks are executed on every push and pull request:

\begin{itemize}
  \item \textbf{Type Check:} TypeScript type checking ensures type safety across the Turborepo monorepo.
  \item \textbf{Lint:} ESLint runs to enforce code quality standards and catch potential errors.
  \item \textbf{Format Check:} Prettier formatting validation ensures consistent code style. The check fails if files are not properly formatted, prompting developers to run the formatter.
  \item \textbf{Build:} Full project build verification ensures all components compile successfully.
  \item \textbf{Test:} Jest unit tests are executed with a PostgreSQL database service container. The test suite includes database schema migrations via Drizzle ORM before running tests.
\end{itemize}

All code quality checks must pass before a pull request can be merged into the main branch, guaranteeing that only validated, production-ready code enters the repository.

\subsubsection{Documentation Quality Checks}

% Source: .github/workflows/ci.yml (lines 115-170): LaTeX quality check jobs (latex-lint, latex-format-check, latex-compile)

For LaTeX documentation changes, the pipeline performs specialized checks:

\begin{itemize}
  \item \textbf{LaTeX Lint:} ChkTeX analyzes LaTeX files for common errors and style issues.
  \item \textbf{LaTeX Format Check:} \texttt{latexindent} validates that LaTeX files follow consistent formatting standards.
  \item \textbf{LaTeX Compilation:} All LaTeX documents are compiled across multiple directories (Checklists, Design documents, SRS, VnV documents, project management documents, etc.) using a matrix strategy. The compilation process handles both simple documents and documents requiring BibTeX bibliography processing.
\end{itemize}

\subsection{Continuous Deployment}

% Source: .github/workflows/buildtex.yml (lines 1-58): Automated PDF compilation and commit workflow

The \texttt{buildtex.yml} workflow handles automated PDF generation for LaTeX documents. When LaTeX files are modified and merged to the main branch, the workflow:

\begin{itemize}
  \item Detects changed \texttt{.tex} files in the \texttt{docs/} directory.
  \item Compiles each changed document using the full TeXLive distribution.
  \item Automatically commits the generated PDF files back to the repository.
\end{itemize}

This ensures that PDF documentation is always up-to-date with the latest LaTeX source files without requiring manual compilation.

\subsection{Design Considerations}

% Source: .github/workflows/ci.yml (lines 9-11, 14-35, 177-192): Concurrency control, path filtering, matrix strategy
% Source: docs/DevelopmentPlan/DevelopmentPlan.tex (lines 114-115): CI/CD design goals

The CI/CD pipeline emphasizes speed, consistency, and modularity:

\begin{itemize}
  \item \textbf{Concurrency Control:} Workflows use GitHub's concurrency groups to cancel in-progress runs when new commits are pushed, preventing unnecessary resource consumption.
  \item \textbf{Path-Based Filtering:} Change detection ensures that code checks only run when code files change, and LaTeX checks only run when documentation changes, significantly reducing build times.
  \item \textbf{Matrix Strategy:} LaTeX compilation uses a matrix to compile documents in parallel across multiple directories, improving efficiency.
  \item \textbf{Fail-Fast Strategy:} The pipeline uses a final aggregation job that checks all individual job results, providing clear feedback on which checks failed.
\end{itemize}

This automated pipeline maintains project stability and code quality as new features and documentation are introduced, while minimizing developer friction through intelligent optimization.
