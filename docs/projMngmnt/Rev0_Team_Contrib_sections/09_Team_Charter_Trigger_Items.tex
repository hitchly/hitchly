% Sources:
% - docs/DevelopmentPlan/DevelopmentPlan.tex (lines 399-406): Quantified metrics (attendance, commits, issues, documentation/reviews)
% - docs/DevelopmentPlan/DevelopmentPlan.tex (lines 415-421): Consequences for violations
% - docs/projMngmnt/POC_Team_Contrib.tex (lines 179-181): Reference to trigger items and absence policy

\section{Team Charter Trigger Items}

\subsection{Summary of Quantified Triggers}

The team charter, as documented in the Development Plan, establishes the following quantified metrics to ensure accountability and consistent contribution from all team members:

\begin{itemize}
    \item \textbf{Attendance:} Members are expected to attend at least 90\% of scheduled team meetings, supervisor/stakeholder meetings, and lectures.
    \item \textbf{Commits:} Each member should make regular commits reflecting steady progress, with a minimum of 2 commits per week per person.
    \item \textbf{Issues:} All members must contribute to opening, discussing, and closing GitHub Issues, with at least 1--2 issues closed per week per person.
    \item \textbf{Documentation \& Reviews:} Members are expected to contribute to documentation and review teammates' work at least once per deliverable.
    \item \textbf{Unexcused Absences:} No more than 2 unexcused absences per term are permitted.
\end{itemize}

These triggers are designed to ensure balanced workload distribution, consistent progress, and high-quality deliverables throughout the project lifecycle.

\subsection{Violations}

% Source: docs/projMngmnt/Rev0_Team_Contrib_sections/05_TA_Document_Discussion_Attendance.tex (line 10): Example of excused absence
% Source: Attendance tables in sections 2-5: All members meeting attendance thresholds

During the Revision 0 period, the team has not experienced any violations of the established triggers. All team members have:

\begin{itemize}
    \item Maintained attendance above the 90\% threshold for all meeting types (team meetings, supervisor meetings, lectures, and TA document discussions).
    \item Made regular commits to the repository, contributing to the project's steady progress.
    \item Actively participated in issue creation, assignment, and resolution.
    \item Engaged in code and documentation reviews as part of the pull request process.
\end{itemize}

Any absences that occurred were communicated to the team in advance with valid reasons, ensuring transparency and maintaining team cohesion. For example, Burhanuddin Kharodawala was unable to attend one TA document discussion session due to being out of town, and this absence was communicated to the team beforehand.

\subsection{Plan to Address Future Violations}

% Source: docs/DevelopmentPlan/DevelopmentPlan.tex (lines 415-421): Progressive consequence structure

The team charter outlines a progressive consequence structure for addressing violations:

\begin{itemize}
    \item \textbf{Initial Response:} Members falling short of expectations will first receive a reminder and the opportunity to catch up on missed contributions.
    \item \textbf{Escalation:} If underperformance persists, the issue will be discussed with the TA or instructor and reflected in peer evaluations.
    \item \textbf{Minor Infractions:} For minor misses (e.g., attendance or commits), light-hearted consequences may be applied, such as bringing coffee or snacks to the next meeting, maintaining a positive team culture while reinforcing expectations.
\end{itemize}

Throughout the project, the team will continue to assess the triggers and modify them as appropriate if they are found to be too weak, too strong, or ambiguous. The goal is to maintain realistic expectations that promote consistent contribution while allowing for flexibility when team members face legitimate challenges.
