\documentclass{article}

\usepackage{booktabs}
\usepackage{tabularx}

\title{Development Plan\\\progname}

\author{\authname}

\date{}

%% Comments

\usepackage{color}

\newif\ifcomments\commentstrue %displays comments
%\newif\ifcomments\commentsfalse %so that comments do not display

\ifcomments
\newcommand{\authornote}[3]{\textcolor{#1}{[#3 ---#2]}}
\newcommand{\todo}[1]{\textcolor{red}{[TODO: #1]}}
\else
\newcommand{\authornote}[3]{}
\newcommand{\todo}[1]{}
\fi

\newcommand{\wss}[1]{\authornote{magenta}{SS}{#1}} 
\newcommand{\plt}[1]{\authornote{cyan}{TPLT}{#1}} %For explanation of the template
\newcommand{\an}[1]{\authornote{cyan}{Author}{#1}}

%% Common Parts

\newcommand{\progname}{ProgName} % PUT YOUR PROGRAM NAME HERE
\newcommand{\authname}{Team \#, Team Name
\\ Student 1 name
\\ Student 2 name
\\ Student 3 name
\\ Student 4 name} % AUTHOR NAMES                  

\usepackage{hyperref}
    \hypersetup{colorlinks=true, linkcolor=blue, citecolor=blue, filecolor=blue,
                urlcolor=blue, unicode=false}
    \urlstyle{same}
                                


\begin{document}

\maketitle

\begin{table}[hp]
\caption{Revision History} \label{TblRevisionHistory}
\begin{tabularx}{\textwidth}{llX}
\toprule
\textbf{Date} & \textbf{Developer(s)} & \textbf{Change}\\
\midrule
September 19th & Swesan Pathmanathan & Added content to Appendix: Reflection and Team Charter\\
Date2 & Name(s) & Description of changes\\
... & ... & ...\\
\bottomrule
\end{tabularx}
\end{table}

\newpage{}
This development plan summarizes the goals, scope, and delivery approach for Hitchly, a cross-platform mobile ridesharing application for the McMaster community. It documents confidentiality and IP considerations, team roles and meeting practices, workflow and CI/CD conventions, the chosen technology stack, and the proof-of-concept priorities (McMaster email verification and a ride-matching prototype). The plan also specifies coding standards, testing strategy, scheduling guidance, and appendices (reflection and team charter) to guide implementation and project governance.

\wss{Additional information on the development plan can be found in the
\href{https://gitlab.cas.mcmaster.ca/courses/capstone/-/blob/main/Lectures/L02b_POCAndDevPlan/POCAndDevPlan.pdf?ref_type=heads}
{lecture slides}.}

\section{Confidential Information?}

\wss{State whether your project has confidential information from industry, or
not.  If there is confidential information, point to the agreement you have in
place.}

\wss{For most teams this section will just state that there is no confidential
information to protect.}
\section{IP to Protect}

\wss{State whether there is IP to protect.  If there is, point to the agreement.
All students who are working on a project that requires an IP agreement are also
required to sign the ``Intellectual Property Guide Acknowledgement.''}

\section{Copyright License}

\wss{What copyright license is your team adopting.  Point to the license in your
repo.}

\section{Team Meeting Plan}

\wss{How often will you meet? where?}

\wss{If the meeting is a physical location (not virtual), out of an abundance of
caution for safety reasons you shouldn't put the location online}

\wss{How often will you meet with your industry advisor?  when?  where?}

\wss{Will meetings be virtual?  At least some meetings should likely be
in-person.}

\wss{How will the meetings be structured?  There should be a chair for all meetings.  There should be an agenda for all meetings.}

\section{Team Communication Plan}

\wss{Issues on GitHub should be part of your communication plan.}

\section{Team Member Roles}

\wss{You should identify the types of roles you anticipate, like notetaker,
leader, meeting chair, reviewer.  Assigning specific people to those roles is
not necessary at this stage.  In a student team the role of the individuals will
likely change throughout the year.}

\section{Workflow Plan}

\begin{itemize}
	\item How will you be using git, including branches, pull request, etc.?
	\item How will you be managing issues, including template issues, issue
	classification, etc.?
  \item Use of CI/CD
\end{itemize}

\section{Project Decomposition and Scheduling}

\begin{itemize}
  \item How will you be using GitHub projects?
  \item Include a link to your GitHub project
\end{itemize}

\wss{How will the project be scheduled?  This is the big picture schedule, not
details. You will need to reproduce information that is in the course outline
for deadlines.}

\section{Proof of Concept Demonstration Plan}

% What is the main risk, or risks, for the success of your project?  What will you
% demonstrate during your proof of concept demonstration to convince yourself that
% you will be able to overcome this risk?

The primary risks for this project are user trust and safety and the reliability of the ride-matching system.
Without a secure verification process tied to McMaster email accounts, users may hesitate to adopt the platform.
Similarly, if the matching algorithm fails to correctly pair riders and drivers based on schedules and locations, the application will not provide value.
The proof of concept will focus on validating these two core capabilities:

\begin{itemize}
  \item \textbf{McMaster Email Verification}
  \begin{itemize}
      \item Users must sign up using a McMaster-affiliated email.
      \item Verification ensures that only students, staff, and faculty can access the platform.
      \item Demonstrates credibility and mitigates safety concerns for early users.
  \end{itemize}

  \item \textbf{Ride Matching Prototype}
  \begin{itemize}
      \item A minimal matching engine takes sample schedules and trip data from riders and drivers.
      \item Produces correct pairings with cost-sharing estimates.
      \item Confirms feasibility of integrating timetable and location-based logic.
  \end{itemize}
\end{itemize}

By validating these areas first, the team ensures the platform addresses its two biggest risks of safety and usability before expanding into additional features such as live tracking, queuing, or ratings.


\section{Expected Technology}

% \wss{What programming language or languages do you expect to use?  What external
% libraries?  What frameworks?  What technologies.  Are there major components of
% the implementation that you expect you will implement, despite the existence of
% libraries that provide the required functionality.  For projects with machine
% learning, will you use pre-trained models, or be training your own model?  }

% \wss{The implementation decisions can, and likely will, change over the course
% of the project.  The initial documentation should be written in an abstract way;
% it should be agnostic of the implementation choices, unless the implementation
% choices are project constraints.  However, recording our initial thoughts on
% implementation helps understand the challenge level and feasibility of a
% project.  It may also help with early identification of areas where project
% members will need to augment their training.}

% Topics to discuss include the following:

% \begin{itemize}
% \item Specific programming language
% \item Specific libraries
% \item Pre-trained models
% \item Specific linter tool (if appropriate)
% \item Specific unit testing framework
% \item Investigation of code coverage measuring tools
% \item Specific plans for Continuous Integration (CI), or an explanation that CI
%   is not being done
% \item Specific performance measuring tools (like Valgrind), if
%   appropriate
% \item Tools you will likely be using?
% \end{itemize}

% \wss{git, GitHub and GitHub projects should be part of your technology.}

The project will be built as a \textbf{cross-platform mobile application} for iOS and Android. The technology stack covers the full lifecycle: ideation, design, development, testing, deployment, and app store release.

\subsection*{a. Concept Ideation \& Design}
\begin{itemize}
    \item \textbf{Wireframing \& Prototyping:} Figma for user flows, wireframes, and visual design.
    \item \textbf{Collaboration \& Task Management:} GitHub Projects for task tracking, milestones, and agile planning.
\end{itemize}

\subsection*{b. Frontend (Mobile App)}
\begin{itemize}
    \item \textbf{Framework:} React Native with Expo (cross-platform).
    \item \textbf{Language:} TypeScript for static typing and type safety.
    \item \textbf{UI \& Styling:} Tailwind CSS via NativeWind and shadcn/ui for consistent styling.
    \item \textbf{Navigation:} React Navigation for multi-screen flows.
    \item \textbf{State Management \& Data Fetching:} React Query integrated with tRPC client for type-safe server communication.
\end{itemize}

\subsection*{c. Backend \& API}
\begin{itemize}
    \item \textbf{Framework:} tRPC running on Node.js (standalone server serving the mobile app).
    \item \textbf{Language:} TypeScript (shared between frontend and backend).
    \item \textbf{Hosting:} Railway for backend server and PostgreSQL database (low cost, minimal maintenance).
    \item \textbf{Authentication:} 
    \begin{itemize}
        \item McMaster email domain verification at signup.
        \item JWT for secure session handling.
    \end{itemize}
    \item \textbf{Core Services:} Matching engine, ride scheduling, and trip storage.
\end{itemize}

\subsection*{d. Database \& Persistence}
\begin{itemize}
    \item \textbf{Database:} PostgreSQL for structured relational data.
    \item \textbf{ORM:} Drizzle ORM for schema management and migrations.
    \item \textbf{Hosting:} Managed PostgreSQL via Railway.
\end{itemize}

\subsection*{e. Testing \& Quality Assurance}
\begin{itemize}
    \item \textbf{Unit Testing:} Jest for frontend and backend logic.
    \item \textbf{Integration Testing:} Supertest for backend endpoint verification.
    \item \textbf{Static Analysis:} ESLint and Prettier for linting and formatting.
\end{itemize}

\subsection*{f. Development Workflow \& CI/CD}
\begin{itemize}
    \item \textbf{Version Control:} Git + GitHub repository.
    \item \textbf{Branching Strategy:} Feature branches with pull requests.
    \item \textbf{CI/CD:} GitHub Actions for automated testing, linting, type-checking, and deployments.
    \item \textbf{Environments:}
    \begin{itemize}
        \item Development: Local Expo Go + Railway database.
        \item Staging: Railway preview deployment + Expo EAS preview builds.
        \item Production: Stable backend deployments + app store builds.
    \end{itemize}
\end{itemize}

\subsection*{g. Deployment \& Hosting}
\begin{itemize}
    \item \textbf{Backend Deployment:} Railway containerized Node.js hosting.
    \item \textbf{Database Deployment:} Railway managed PostgreSQL.
    \item \textbf{Frontend Builds:} Expo EAS Build for iOS and Android binaries.
    \item \textbf{App Store Distribution:}
    \begin{itemize}
        \item Apple App Store via Expo EAS.
        \item Google Play Store via Expo EAS.
    \end{itemize}
\end{itemize}


\section{Coding Standard}

% \wss{What coding standard will you adopt?}

To maintain \textbf{clarity, maintainability, and collaboration}, the team will adopt the following coding standards:

\subsection*{Style \& Formatting}
\begin{itemize}
    \item TypeScript with consistent naming conventions and modular architecture.
    \item ESLint and Prettier enforce linting and formatting across the codebase.
\end{itemize}

\subsection*{Documentation}
\begin{itemize}
    \item JSDoc comments for functions, APIs, and modules.
    \item Inline comments for complex logic (e.g., ride-matching algorithm).
\end{itemize}

\subsection*{Version Control Practices}
\begin{itemize}
    \item Git with feature branches for all new work.
    \item Pull requests with mandatory peer review before merging.
    \item All commits to main branch must be squashed and merged (enforced via GitHub rulesets).
    \item Descriptive commit messages using Conventional Commits format.
    \item GitHub Projects for issue tracking, milestones, and labeling (bug, feature, documentation, etc.).
\end{itemize}

\subsection*{Testing Practices}
\begin{itemize}
    \item Unit tests for core features (authentication, ride matching).
    \item Integration tests for backend endpoints.
\end{itemize}

These standards ensure the codebase is \textbf{professional, maintainable, and scalable}, meeting rubric expectations for a senior design project.

\newpage{}

\section*{Appendix --- Reflection}

The purpose of reflection questions is to give you a chance to assess your own
learning and that of your group as a whole, and to find ways to improve in the
future. Reflection is an important part of the learning process.  Reflection is
also an essential component of a successful software development process.  

Reflections are most interesting and useful when they're honest, even if the
stories they tell are imperfect. You will be marked based on your depth of
thought and analysis, and not based on the content of the reflections
themselves. Thus, for full marks we encourage you to answer openly and honestly
and to avoid simply writing ``what you think the evaluator wants to hear.''

Please answer the following questions.  Some questions can be answered on the
team level, but where appropriate, each team member should write their own
response:


\begin{enumerate}
  \item \textbf{Why is it important to create a development plan prior to starting the project?} \\ 
  Creating a development plan is important because it establishes structure, expectations, and direction for the team before work begins. It ensures that roles, responsibilities, communication methods, and goals are clearly defined, which reduces confusion later in the project. It also helps identify risks early and provides a roadmap that keeps the team accountable and aligned.
  \item \textbf{In your opinion, what are the advantages and disadvantages of using CI/CD?} \\ 
  The main advantages of CI/CD are faster development cycles, early detection of bugs, and consistent integration of code, which reduces the likelihood of major conflicts at later stages. It also enforces good coding practices and increases overall software reliability. However, disadvantages include the setup time and learning curve required, especially for a student project, as well as the overhead of maintaining pipelines and infrastructure when the team may already be managing heavy coursework.
  \item \textbf{What disagreements did your group have in this deliverable, if any, and how did you resolve them?} \\ 
  Not in this deliverable in particular, but earlier we had a disagreement regarding the proposal idea for our capstone project. After debating with facts and reasoning, we collectively concluded that pursuing the rideshare project would be the most impactful and feasible direction for our team.
\end{enumerate}

\newpage{}

\section*{Appendix --- Team Charter}

\subsection*{External Goals}

Our external goals for this project are to develop an application that delivers meaningful real-world impact that also extends beyond the scope of capstone. We aim to achieve a high grade in this course, higher than a 10, while also competing at the Capstone EXPO and presenting our work to the McMaster community. In addition, we intend for this project to serve as a strong portfolio piece, highlighting our technical, design, and collaborative skills to future employers. Finally, we recognize the potential for this project to evolve into a startup, enabling it to create value and impact beyond the university setting—potentially the next big ride app competing against Uber and Lyft.

\subsection*{Attendance}

\subsubsection*{Expectations}

Our team expects all members to treat meetings as a priority and respect one another’s time by being punctual and engaged. Attendance is essential to keep the project on track, and clear communication is expected whenever conflicts arise.

\begin{itemize}
    \item Members should arrive on time; being more than 10 minutes late without notice is discouraged, as it is disruptive and shows a lack of respect for everyone else’s time.
    \item No more than 2 unexcused absences are permitted per term.
    \item Leaving early or joining late is only acceptable if communicated beforehand.
    \item Members are expected to be fully present during meetings and avoid unrelated activities.
    \item Anyone who misses a meeting is responsible for reviewing notes and following up to stay up to date.
\end{itemize}

\subsubsection*{Acceptable Excuse}

\begin{itemize}
    \item Transit delays, weather issues, or unforeseen personal emergencies (as long as the team is notified).
    \item Illness, family emergencies, or direct academic conflicts such as exams or required presentations.
    \item Overlapping academic or work responsibilities, or unavoidable scheduling conflicts that are shared in advance.
    \item Urgent situations (e.g., emergency calls or messages) that require brief attention, but members should minimize distractions.
\end{itemize}

\textbf{Note:} There is no acceptable excuse for failing to review notes and catch up after missing a meeting — this is always the member’s responsibility.

\subsubsection*{In Case of Emergency}

If a team member experiences an emergency and cannot attend a meeting or complete their assigned work:
\begin{itemize}
    \item They must notify the team as soon as possible through the primary communication channel (e.g., Discord/Teams).
    \item No personal details need to be shared beyond confirming that it is an emergency.
    \item The member should communicate how much of their work has been completed and what remains outstanding.
    \item If the task cannot be finished in time, the member should either:
    \begin{itemize}
        \item Delay completion and commit to catching up later, or
        \item Transfer responsibility to another team member, providing enough context or materials to allow for a smooth handoff.
    \end{itemize}
    \item If responsibility is transferred, the original member should agree on how they will make up their contribution in a future task to keep workloads fair.
\end{itemize}


\subsection*{Accountability and Teamwork}

\subsubsection*{Quality}

Our team expects all members to come prepared for meetings, with assigned tasks completed to a high standard. Deliverables should be accurate, polished, and ready for integration or review. Members are encouraged to review one another’s work to ensure consistency and to help catch errors early. The expectation is to go beyond simply completing tasks, aiming instead to produce work that reflects professionalism and attention to detail.

\subsubsection*{Attitude}

Team members are expected to maintain a positive and professional attitude, respecting all contributions and encouraging open discussion. All ideas will be heard and considered, but discussion should remain solutions focused. Disagreements will be handled constructively, using a clear process: clarify the issue, listen to all perspectives, establish common ground, and agree on a resolution.

\subsubsection*{Stay on Track}

Our team will stay accountable by setting measurable targets and monitoring progress regularly. Each member is expected to contribute fairly and consistently so that the workload is balanced, and deadlines are met.

\vspace{0.5em}
\noindent\textbf{Metrics:}
\begin{itemize}
    \item \textbf{Attendance:} Members are expected to attend at least 90\% of scheduled meetings.
    \item \textbf{Commits:} Each member should make regular commits (minimum 2 per week), reflecting steady progress.
    \item \textbf{Issues:} All members must contribute to opening, discussing, and closing GitHub Issues, with at least 1--2 issues closed per week per person.
    \item \textbf{Documentation \& Reviews:} Members are expected to contribute to documentation and review teammates’ work at least once per deliverable.
\end{itemize}

\vspace{0.5em}
\noindent\textbf{Rewards:}
\begin{itemize}
    \item The top contributor for each milestone (measured by commits, issues closed, and meeting contributions) will be recognized and treated to a free lunch by the team.
    \item Members who consistently meet or exceed expectations may receive lighter workloads during peak deadlines as recognition.
\end{itemize}

\vspace{0.5em}
\noindent\textbf{Consequences:}
\begin{itemize}
    \item Members falling short of expectations will first receive a reminder and the chance to catch up.
    \item If underperformance persists, the issue will be discussed with the TA or instructor and reflected in peer evaluations.
    \item For minor misses (e.g., attendance or commits), fun consequences may be applied, such as bringing coffee/snacks to the next meeting.
\end{itemize}
\subsubsection*{Team Building}

Cohesion will be maintained through open communication, supportive collaboration, and celebrating milestones together. Occasional informal check-ins will help strengthen relationships and ensure a positive team culture.

\subsubsection*{Decision Making}

Decisions will aim for consensus whenever possible. If consensus cannot be reached, the team will rely on rational, objective, and fact-based decision making rather than subjective opinions. Evidence, data, and alignment with project goals will guide choices. As a last resort, majority voting will be used.

\end{document}