\documentclass[12pt, titlepage]{article}

\usepackage{fullpage}
\usepackage[round]{natbib}
\usepackage{multirow}
\usepackage{booktabs}
\usepackage{tabularx}
\usepackage{graphicx}
\usepackage{float}
\usepackage{hyperref}
\hypersetup{
    colorlinks,
    citecolor=blue,
    filecolor=black,
    linkcolor=red,
    urlcolor=blue
}

\input{../../Comments}
%% Common Parts

\newcommand{\progname}{Software Engineering} % PUT YOUR PROGRAM NAME HERE
\newcommand{\authname}{Team \#16, The Chill Guys
\\ Hamzah Rawasia
\\ Sarim Zia
\\ Aidan Froggatt
\\ Swesan Pathmanathan
\\ Burhanuddin Kharodawala} % AUTHOR NAMES                  

\usepackage{hyperref}
    \hypersetup{colorlinks=true, linkcolor=blue, citecolor=blue, filecolor=blue,
                urlcolor=blue, unicode=false}
    \urlstyle{same}
                                


\newcounter{acnum}
\newcommand{\actheacnum}{AC\theacnum}
\newcommand{\acref}[1]{AC\ref{#1}}

\newcounter{ucnum}
\newcommand{\uctheucnum}{UC\theucnum}
\newcommand{\uref}[1]{UC\ref{#1}}

\newcounter{mnum}
\newcommand{\mthemnum}{M\themnum}
\newcommand{\mref}[1]{M\ref{#1}}

\begin{document}

\title{Module Guide for \progname{}} 
\author{\authname}
\date{\today}

\maketitle

\pagenumbering{roman}

\section{Revision History}

\begin{tabularx}{\textwidth}{p{3cm}p{2cm}X}
\toprule {\bf Date} & {\bf Version} & {\bf Notes}\\
\midrule
Date 1 & 1.0 & Notes\\
Date 2 & 1.1 & Notes\\
\bottomrule
\end{tabularx}

\newpage

\section{Reference Material}

This section records information for easy reference.

\subsection{Abbreviations and Acronyms}

\renewcommand{\arraystretch}{1.2}
\begin{tabular}{l l} 
  \toprule		
  \textbf{symbol} & \textbf{description}\\
  \midrule 
  AC & Anticipated Change\\
  DAG & Directed Acyclic Graph \\
  M & Module \\
  MG & Module Guide \\
  OS & Operating System \\
  R & Requirement\\
  SC & Scientific Computing \\
  SRS & Software Requirements Specification\\
  \progname & Explanation of program name\\
  UC & Unlikely Change \\
  \bottomrule
\end{tabular}\\

\newpage

\tableofcontents

\listoftables

\listoffigures

\newpage

\pagenumbering{arabic}

\section{Introduction}

Decomposing a system into modules is a commonly accepted approach to developing
software.  A module is a work assignment for a programmer or programming
team~\citep{ParnasEtAl1984}.  We advocate a decomposition
based on the principle of information hiding~\citep{Parnas1972a}.  This
principle supports design for change, because the ``secrets'' that each module
hides represent likely future changes.  Design for change is valuable in SC,
where modifications are frequent, especially during initial development as the
solution space is explored.  

Our design follows the rules layed out by \citet{ParnasEtAl1984}, as follows:
\begin{itemize}
\item System details that are likely to change independently should be the
  secrets of separate modules.
\item Each data structure is implemented in only one module.
\item Any other program that requires information stored in a module's data
  structures must obtain it by calling access programs belonging to that module.
\end{itemize}

After completing the first stage of the design, the Software Requirements
Specification (SRS), the Module Guide (MG) is developed~\citep{ParnasEtAl1984}. The MG
specifies the modular structure of the system and is intended to allow both
designers and maintainers to easily identify the parts of the software.  The
potential readers of this document are as follows:

\begin{itemize}
\item New project members: This document can be a guide for a new project member
  to easily understand the overall structure and quickly find the
  relevant modules they are searching for.
\item Maintainers: The hierarchical structure of the module guide improves the
  maintainers' understanding when they need to make changes to the system. It is
  important for a maintainer to update the relevant sections of the document
  after changes have been made.
\item Designers: Once the module guide has been written, it can be used to
  check for consistency, feasibility, and flexibility. Designers can verify the
  system in various ways, such as consistency among modules, feasibility of the
  decomposition, and flexibility of the design.
\end{itemize}

The rest of the document is organized as follows. Section
\ref{SecChange} lists the anticipated and unlikely changes of the software
requirements. Section \ref{SecMH} summarizes the module decomposition that
was constructed according to the likely changes. Section \ref{SecConnection}
specifies the connections between the software requirements and the
modules. Section \ref{SecMD} gives a detailed description of the
modules. Section \ref{SecTM} includes two traceability matrices. One checks
the completeness of the design against the requirements provided in the SRS. The
other shows the relation between anticipated changes and the modules. Section
\ref{SecUse} describes the use relation between modules.

\section{Anticipated and Unlikely Changes} \label{SecChange}

This section lists possible changes to the system. According to the likeliness
of the change, the possible changes are classified into two
categories. Anticipated changes are listed in Section \ref{SecAchange}, and
unlikely changes are listed in Section \ref{SecUchange}.

\subsection{Anticipated Changes} \label{SecAchange}

Anticipated changes are the source of the information that is to be hidden
inside the modules. Ideally, changing one of the anticipated changes will only
require changing the one module that hides the associated decision. The approach
adapted here is called design for
change.

\begin{description}
\item[\refstepcounter{acnum} \actheacnum \label{acMatchingAlgorithm}:] \textbf{Matching Algorithm:} The matching algorithm focuses on 
the goal of finding the best available drivers for the riders. This algorithm takes into account the user's location, timetable, time, and distance. For instance, it matches a rider with a driver that has the same timetable and lives in the same area. The current algorithm will be simply based on these four parameters, and it will have to change in the future to consider more parameters to increase its efficiency and accuracy. Moreover, subtle changes to the algorithm may also be expected for further optimization of its complexity and speed. 
\item[\refstepcounter{acnum} \actheacnum \label{acUserInterface}:] \textbf{User Interface:} As the application proceeds to go through its initial phase of development and use, usability testing is expected to be done to ensure user satisfaction. As the users provide feedback, changes will be made to the user interface in the future phases of development. Moreover, even after that, changes will be made based on accessibility requirements in the future. The layout and navigation are the two main things from the user interaction that would need the most changes as per user's feedback.   
\item[\refstepcounter{acnum} \actheacnum \label{acDB}:] \textbf{Database Schema:} As the app updates with new features, the current method of analytics may change. New parameters would be considered for the matching algorithm, and new features may be added based on future user needs. Therefore, changes may need to be made to the database schemas
\item[\refstepcounter{acnum} \actheacnum \label{acPayment}:] \textbf{Payment Processing:} Additional payment processors and pricing formulas will be used to determine the price estimate for drivers. An addition of parameters to estimate the cost would cause the algorithms to change.  
\item[\refstepcounter{acnum} \actheacnum \label{acDevEnvioremnt}:] \textbf{Development Environment:} Changes could be made to the CI/CD configuration to improve the reliability and stability of the system.
\end{description}


\subsection{Unlikely Changes} \label{SecUchange}

The module design should be as general as possible. However, a general system is
more complex. Sometimes this complexity is not necessary. Fixing some design
decisions at the system architecture stage can simplify the software design. If
these decision should later need to be changed, then many parts of the design
will potentially need to be modified. Hence, it is not intended that these
decisions will be changed.

\begin{description}
\item[\refstepcounter{ucnum} \uctheucnum \label{ucDevices}:] \textbf{Devices:} Hitchly is an application that is developed for mobile devices only. A version that is compatible to run on a desktop is an unlikely change as it requires major changes to the software design. 
\item[\refstepcounter{ucnum} \uctheucnum \label{ucLanguage}:] \textbf{Language Support:} The application is primarily based on the English language. Adding additional languages to the interface would require a major design decision.
\item[\refstepcounter{ucnum} \uctheucnum \label{ucProgramming}:] \textbf{Programming Language and Technologies:} The core of this application is supposed to be developed using React Native, Typescript, and SQL. Making changes to how the application is developed is an unlikely change as it requires a change in the entire design of the application.
\item[\refstepcounter{ucnum} \uctheucnum \label{ucVerification}:] \textbf{Verification Requirement:} This app is strictly restricted for McMaster affiliated users as the entire concept of this application is designed for them. It is not subject to removal. Verification of the users will be their McMaster affiliated email.
\end{description}

\section{Module Hierarchy} \label{SecMH}

This section provides an overview of the modular design for Hitchly.  
The hierarchy follows the principle of information hiding, where each module  
encapsulates a design decision that may change independently.  
Only the leaf modules shown in Table~\ref{TblMH} are implemented.

\begin{description}

\item[\textbf{5.1 Hardware-Hiding Modules}]  
There are no hardware-hiding modules in this design.  
Hitchly relies on high-level framework APIs (Expo, React Native),  
so no direct hardware abstraction is required.

\item[\textbf{5.2 Behaviour-Hiding Modules}]  
These modules implement the user-visible behaviour of the application.  
They hide workflow logic, API interactions, and data handling that support  
Hitchly’s core features.

\begin{itemize}
  \item M1: Authentication \& Verification Module
  \item M2: User Profile Module
  \item M3: Route \& Trip Module
  \item M4: Matching Module
  \item M5: Scheduling Module
  \item M6: Notification Module
  \item M7: Rating \& Feedback Module
  \item M8: Safety \& Reporting Module
  \item M9: Payment \& Cost Estimation Module
  \item M10: Admin \& Moderation Module
\end{itemize}

\item[\textbf{5.3 Software-Decision Modules}]  
These modules encapsulate internal logic, algorithms,  
and implementation decisions not visible to users.

\begin{itemize}
  \item M11: Pricing Module
  \item M12: Database Module
\end{itemize}

\end{description}

\begin{table}[h!]
\centering
\begin{tabular}{p{0.3\textwidth} p{0.6\textwidth}}
\toprule
\textbf{Level 1} & \textbf{Level 2}\\
\midrule
Hardware-Hiding Module & None\\
\midrule
\multirow{10}{0.3\textwidth}{Behaviour-Hiding Module}
& M1: Authentication \& Verification Module\\
& M2: User Profile Module\\
& M3: Route \& Trip Module\\
& M4: Matching Module\\
& M5: Scheduling Module\\
& M6: Notification Module\\
& M7: Rating \& Feedback Module\\
& M8: Safety \& Reporting Module\\
& M9: Payment \& Cost Estimation Module\\
& M10: Admin \& Moderation Module\\
\midrule
\multirow{2}{0.3\textwidth}{Software-Decision Module}
& M11: Pricing Module\\
& M12: Database Module\\
\bottomrule
\end{tabular}
\caption{Module Hierarchy}
\label{TblMH}
\end{table}



\section{Connection Between Requirements and Design} \label{SecConnection}

The design of the system is intended to satisfy the requirements developed in
the SRS. In this stage, the system is decomposed into modules. The connection
between requirements and modules is listed in Table~\ref{TblRT}.

\section{Module Decomposition} \label{SecMD}

\begingroup
\setlength{\parindent}{0pt}

Modules are decomposed according to the principle of information hiding
\citep{ParnasEtAl1984}.  
Each module’s \textit{Secrets} describes the internal design decision that is
intentionally hidden from other modules, while the \textit{Services} field specifies
\textit{what} the module provides without revealing \textit{how} the service is implemented.  
The \textit{Frontend UI}, \textit{API Logic}, and \textit{Database Models} indicate 
how each module is realized across the system’s architecture.  
\textit{Implemented By} identifies the technologies used, and 
\textit{Type of Module} classifies each module as a Behaviour-Hiding or 
Software-Decision module according to Section~\ref{SecMH}.  
Only the leaf modules are included.

% ---------------------------------------------------------
% BEHAVIOUR-HIDING MODULES
% ---------------------------------------------------------
\subsection*{Behaviour-Hiding Modules}

\subsubsection*{M1: Authentication \& Verification Module}
\begin{itemize}
  \item \textbf{Secrets:} Token/session strategy, email-verification workflow, and credential-validation logic.
  \item \textbf{Services:} Allows users to register, log in, and verify McMaster email accounts.
  \item \textbf{Frontend UI:} Login screens, signup forms, verification code UI.
  \item \textbf{API Logic:} \texttt{authRouter} (login, register, verify), session helpers, Zod validation.
  \item \textbf{Database Models:} \texttt{User} table (email, password hash, verified flag, session tokens).
  \item \textbf{Implemented By:} React Native (Expo) + tRPC backend with Prisma ORM.
  \item \textbf{Type of Module:} Behaviour-Hiding Module.
\end{itemize}

\subsubsection*{M2: User Profile Module}
\begin{itemize}
  \item \textbf{Secrets:} Role-assignment logic and profile-preference storage design.
  \item \textbf{Services:} Provides profile editing, preferences, and vehicle data management.
  \item \textbf{Frontend UI:} Profile page, edit forms, vehicle info inputs.
  \item \textbf{API Logic:} \texttt{userRouter} for CRUD operations, preference update flow.
  \item \textbf{Database Models:} \texttt{User}, \texttt{Vehicle}, \texttt{Preference}.
  \item \textbf{Implemented By:} React Native UI + tRPC + Prisma.
  \item \textbf{Type of Module:} Behaviour-Hiding Module.
\end{itemize}

\subsubsection*{M3: Route \& Trip Module}
\begin{itemize}
  \item \textbf{Secrets:} Trip-storage schema, route normalization, and time formatting.
  \item \textbf{Services:} Allows users to create, list, and cancel trips.
  \item \textbf{Frontend UI:} Trip creation form, trip list UI, route selection screen.
  \item \textbf{API Logic:} \texttt{tripRouter} for trip creation, querying, and deletion.
  \item \textbf{Database Models:} \texttt{Trip} table (origin, destination, time, seat count).
  \item \textbf{Implemented By:} React Native + tRPC backend + Prisma.
  \item \textbf{Type of Module:} Behaviour-Hiding Module.
\end{itemize}

\subsubsection*{M4: Matching Module}
\begin{itemize}
  \item \textbf{Secrets:} Scoring algorithm, distance/time weighting, and Strategy Pattern implementation.
  \item \textbf{Services:} Computes driver–rider compatibility and returns ordered match results.
  \item \textbf{Frontend UI:} Swipe-based card UI, match results list.
  \item \textbf{API Logic:} \texttt{matchmakingRouter}, MatchEngine (Strategy Pattern).
  \item \textbf{Database Models:} \texttt{Match}, \texttt{Swipe}.
  \item \textbf{Implemented By:} Node.js backend (tRPC) + algorithm utilities + Prisma ORM.
  \item \textbf{Type of Module:} Behaviour-Hiding Module.
\end{itemize}

\subsubsection*{M5: Scheduling Module}
\begin{itemize}
  \item \textbf{Secrets:} Recurring-trip generation algorithm and time-window parsing.
  \item \textbf{Services:} Creates recurring or one-time schedules linked to trips.
  \item \textbf{Frontend UI:} Time picker, recurring toggle, calendar UI.
  \item \textbf{API Logic:} Schedule parser and recurring schedule generator.
  \item \textbf{Database Models:} \texttt{Schedule} table.
  \item \textbf{Implemented By:} tRPC backend + Prisma.
  \item \textbf{Type of Module:} Behaviour-Hiding Module.
\end{itemize}

\subsubsection*{M6: Notification Module}
\begin{itemize}
  \item \textbf{Secrets:} Push-token handling and asynchronous event queue logic.
  \item \textbf{Services:} Sends push notifications for matches, cancellations, and reminders.
  \item \textbf{Frontend UI:} In-app notification center, Expo push integration.
  \item \textbf{API Logic:} Notification service with event emitters.
  \item \textbf{Database Models:} \texttt{Notification} table.
  \item \textbf{Implemented By:} Expo Push Service + tRPC backend dispatcher.
  \item \textbf{Type of Module:} Behaviour-Hiding Module.
\end{itemize}

\subsubsection*{M7: Rating \& Feedback Module}
\begin{itemize}
  \item \textbf{Secrets:} Reputation computation and weighting logic.
  \item \textbf{Services:} Enables users to submit ratings and text feedback after trips.
  \item \textbf{Frontend UI:} Post-ride rating screen.
  \item \textbf{API Logic:} \texttt{ratingRouter}.
  \item \textbf{Database Models:} \texttt{Rating} table (linked to \texttt{Match}).
  \item \textbf{Implemented By:} React Native + tRPC + Prisma ORM.
  \item \textbf{Type of Module:} Behaviour-Hiding Module.
\end{itemize}

\subsubsection*{M8: Safety \& Reporting Module}
\begin{itemize}
  \item \textbf{Secrets:} Incident-flagging thresholds and safety-response workflow.
  \item \textbf{Services:} Allows reporting of unsafe behaviour and stores incidents for admin review.
  \item \textbf{Frontend UI:} Report button, safety resources screen.
  \item \textbf{API Logic:} Report submission via \texttt{reportRouter}.
  \item \textbf{Database Models:} \texttt{Report}.
  \item \textbf{Implemented By:} React Native + tRPC + Prisma.
  \item \textbf{Type of Module:} Behaviour-Hiding Module.
\end{itemize}

\subsubsection*{M9: Payment \& Cost Estimation Module}
\begin{itemize}
  \item \textbf{Secrets:} Fare-calculation logic and mock-payment validation.
  \item \textbf{Services:} Estimates ride cost and records payment confirmations.
  \item \textbf{Frontend UI:} Fare display, confirmation screen.
  \item \textbf{API Logic:} \texttt{paymentRouter}.
  \item \textbf{Database Models:} \texttt{Payment}, \texttt{Transaction}.
  \item \textbf{Implemented By:} tRPC backend + Prisma ORM.
  \item \textbf{Type of Module:} Behaviour-Hiding Module.
\end{itemize}

\subsubsection*{M10: Admin \& Moderation Module}
\begin{itemize}
  \item \textbf{Secrets:} Moderation rules, flagging thresholds, and admin-only access logic.
  \item \textbf{Services:} Allows admins to review reports, ban users, and view platform stats.
  \item \textbf{Frontend UI:} Admin dashboard (optional).
  \item \textbf{API Logic:} \texttt{adminRouter}.
  \item \textbf{Database Models:} Admin logs, flagged users.
  \item \textbf{Implemented By:} tRPC backend + Prisma.
  \item \textbf{Type of Module:} Behaviour-Hiding Module.
\end{itemize}

% ---------------------------------------------------------
% SOFTWARE-DECISION MODULES
% ---------------------------------------------------------
\subsection*{Software-Decision Modules}

\subsubsection*{M11: Pricing Module}
\begin{itemize}
  \item \textbf{Secrets:} Fuel-rate constants, distance multipliers, and fare model.
  \item \textbf{Services:} Computes estimated trip price using distance and conditions.
  \item \textbf{Frontend UI:} Fare estimate displayed in trip preview.
  \item \textbf{API Logic:} Cost-calculation service reused by M9.
  \item \textbf{Database Models:} Pricing configuration constants.
  \item \textbf{Implemented By:} Node.js backend utilities + shared pricing helpers.
  \item \textbf{Type of Module:} Software-Decision Module.
\end{itemize}

\subsubsection*{M12: Database Module}
\begin{itemize}
  \item \textbf{Secrets:} ORM mapping, schema decisions, and migration strategy.
  \item \textbf{Services:} Provides access to all persistent data models via a centralized DB client.
  \item \textbf{Frontend UI:} None.
  \item \textbf{API Logic:} Prisma client configuration and model exports.
  \item \textbf{Database Models:} All tables (User, Trip, Match, Schedule, etc.).
  \item \textbf{Implemented By:} Prisma ORM + PostgreSQL.
  \item \textbf{Type of Module:} Software-Decision Module.
\end{itemize}

\endgroup


\section{Traceability Matrix} \label{SecTM}

This section shows two traceability matrices: between the modules and the
requirements and between the modules and the anticipated changes.

% the table should use mref, the requirements should be named, use something
% like fref
\begin{table}[H]
\centering
\begin{tabular}{p{0.2\textwidth} p{0.6\textwidth}}
\toprule
\textbf{Req.} & \textbf{Modules}\\
\midrule
FR211 & M4, M3, M5, M10\\
FR212 & M4, M5, M10\\
FR213 & M4, M2\\
FR214 & M4, M2, M5\\
NFR211 & M4, M10\\
NFR212 & M4, M1, M10\\
FR221 & M5, M3, M10\\
FR222 & M5, M6\\
FR223 & M5, M6, M9\\
FR224 & M6, M5\\
NFR221 & M10, M5\\
NFR222 & M6, M5, M10\\
FR231 & M1, M2\\
FR232 & M1, M10\\
FR233 & M8, M3, M6\\
FR234 & M8, M2, M12\\
NFR231 & M1, M10\\
NFR232 & M1, M10, M12\\
FR241 & M11, M5, M3\\
FR242 & M9\\
FR243 & M9\\
FR244 & M10, M9\\
NFR241 & M9\\
NFR242 & M9\\
FR251 & M2, M1\\
FR252 & M10, M2\\
FR253 & M10, M2, M3\\
FR254 & M2, M4\\
NFR251 & M2, M1\\
NFR252 & M2, M10\\
FR261 & M7, M2\\
FR262 & M8, M12\\
FR263 & M7, M5, M4\\
FR264 & M7, M2\\
NFR261 & M7, M12\\



\bottomrule
\end{tabular}
\caption{Trace Between Requirements and Modules}
\label{TblRT}
\end{table}

\begin{table}[H]
\centering
\begin{tabular}{p{0.2\textwidth} p{0.6\textwidth}}
\toprule
\textbf{AC} & \textbf{Modules}\\
\midrule
\acref{acMatchingAlgorithm} & M4, M2, M5\\
\acref{acUserInterface} & M1, M2, M3, M4, M5, M6, M7, M8, M9, M11, M12\\
\acref{acDB} & M10\\
\acref{acPayment} & M9, M11\\
\acref{acDevEnvioremnt} & N/A (External Process)\\
\bottomrule
\end{tabular}
\caption{Trace Between Anticipated Changes and Modules}
\label{TblACT}
\end{table}

\section{Use Hierarchy Between Modules} \label{SecUse}

In this section, the uses hierarchy between modules is
provided. \citet{Parnas1978} said of two programs A and B that A {\em uses} B if
correct execution of B may be necessary for A to complete the task described in
its specification. That is, A {\em uses} B if there exist situations in which
the correct functioning of A depends upon the availability of a correct
implementation of B.  Figure \ref{FigUH} illustrates the use relation between
the modules. It can be seen that the graph is a directed acyclic graph
(DAG). Each level of the hierarchy offers a testable and usable subset of the
system, and modules in the higher level of the hierarchy are essentially simpler
because they use modules from the lower levels.

\wss{The uses relation is not a data flow diagram.  In the code there will often
be an import statement in module A when it directly uses module B.  Module B
provides the services that module A needs.  The code for module A needs to be
able to see these services (hence the import statement).  Since the uses
relation is transitive, there is a use relation without an import, but the
arrows in the diagram typically correspond to the presence of import statement.}

\wss{If module A uses module B, the arrow is directed from A to B.}

\begin{figure}[H]
\centering
%\includegraphics[width=0.7\textwidth]{UsesHierarchy.png}
\caption{Use hierarchy among modules}
\label{FigUH}
\end{figure}

%\section*{References}

\section{User Interfaces}

This sections provides an overview of the important screens of the application. 

\begin{figure}[h!]
\caption{Main Login Screen}
\centering
\includegraphics[width=0.2\textwidth]{figure/Login Screen.png}
\end{figure}


\begin{figure}[h!]
\caption{Register Screen}
\centering
\includegraphics[width=0.2\textwidth]{figure/Register Screen.png}
\end{figure}


\begin{figure}[h!]
\caption{Verification Screen}
\centering
\includegraphics[width=0.2\textwidth]{D:\University\4th year\4th_Year_Fall\4G06\hitchly\docs\Design\SoftArchitecture\figure\Register-2.png}
\end{figure}


\begin{figure}[h!]
\caption{Upload Schedule Screen}
\centering
\includegraphics[width=0.2\textwidth]{D:\University\4th year\4th_Year_Fall\4G06\hitchly\docs\Design\SoftArchitecture\figure\Register-3.png}
\end{figure}

\begin{figure}[h!]
\caption{Select User Type Screen}
\centering
\includegraphics[width=0.2\textwidth]{D:\University\4th year\4th_Year_Fall\4G06\hitchly\docs\Design\SoftArchitecture\figure\Register-3 (1).png}
\end{figure}

\begin{figure}[h!]
\caption{User Prompted to Add Profile Information}
\centering
\includegraphics[width=0.2\textwidth]{D:\University\4th year\4th_Year_Fall\4G06\hitchly\docs\Design\SoftArchitecture\figure\Profile.png}
\end{figure}

\begin{figure}[h!]
\caption{Home Screen}
\centering
\includegraphics[width=0.2\textwidth]{D:\University\4th year\4th_Year_Fall\4G06\hitchly\docs\Design\SoftArchitecture\figure\Home-Passenger-Selection (1).png}
\end{figure}

\begin{figure}[h!]
\caption{Switch to Driver Screen}
\centering
\includegraphics[width=0.2\textwidth]{D:\University\4th year\4th_Year_Fall\4G06\hitchly\docs\Design\SoftArchitecture\figure\Home-Drive-Selection.png}
\end{figure}

\begin{figure}[h!]
\caption{Switch to Rider Screen}
\centering
\includegraphics[width=0.2\textwidth]{D:\University\4th year\4th_Year_Fall\4G06\hitchly\docs\Design\SoftArchitecture\figure\Home-Passenger-Selection.png}
\end{figure}

\begin{figure}[h!]
\caption{Driver Connection Screen}
\centering
\includegraphics[width=0.2\textwidth]{D:\University\4th year\4th_Year_Fall\4G06\hitchly\docs\Design\SoftArchitecture\figure\Home-Drive-Connection.png}
\end{figure}


\begin{figure}[h!]
\caption{Rider Connection Screen}
\centering
\includegraphics[width=0.2\textwidth]{D:\University\4th year\4th_Year_Fall\4G06\hitchly\docs\Design\SoftArchitecture\figure\Home-Passenger-Connection.png}
\end{figure}

\begin{figure}[h!]
\caption{Matched Rider List Screen}
\centering
\includegraphics[width=0.2\textwidth]{D:\University\4th year\4th_Year_Fall\4G06\hitchly\docs\Design\SoftArchitecture\figure\Home-Passenger-Connection (2).png}
\end{figure}

\begin{figure}[h!]
\caption{Matched Driver List Screen}
\centering
\includegraphics[width=0.2\textwidth]{D:\University\4th year\4th_Year_Fall\4G06\hitchly\docs\Design\SoftArchitecture\figure\Home-Passenger-Connection (1).png}
\end{figure}

\begin{figure}[h!]
\caption{No Matches Found Screen}
\centering
\includegraphics[width=0.2\textwidth]{D:\University\4th year\4th_Year_Fall\4G06\hitchly\docs\Design\SoftArchitecture\figure\Home-Passenger-Connection (3).png}
\end{figure}

\begin{figure}[h!]
\caption{No Matches Found Screen}
\centering
\includegraphics[width=0.2\textwidth]{D:\University\4th year\4th_Year_Fall\4G06\hitchly\docs\Design\SoftArchitecture\figure\Home-Passenger-Connection (3).png}
\end{figure}

\begin{figure}[h!]
\caption{Driver Navigation Screen}
\centering
\includegraphics[width=0.2\textwidth]{D:\University\4th year\4th_Year_Fall\4G06\hitchly\docs\Design\SoftArchitecture\figure\Driver-navigation.png}
\end{figure}

\begin{figure}[h!]
\caption{Rider Wait Screen}
\centering
\includegraphics[width=0.2\textwidth]{D:\University\4th year\4th_Year_Fall\4G06\hitchly\docs\Design\SoftArchitecture\figure\Passenger-pickup-wait.png}
\end{figure}

\begin{figure}[h!]
\caption{Cost Estimation Screen}
\centering
\includegraphics[width=0.2\textwidth]{D:\University\4th year\4th_Year_Fall\4G06\hitchly\docs\Design\SoftArchitecture\figure\Home-Drive-Connection (1).png}
\end{figure}

\section{Design of Communication Protocols}

This section describes the design of the communication protocols implemented in Hitchly.
Communication occurs primarily between the client‐side mobile application (developed
with React Native / Expo) and the backend server (implemented with Node.js, Express,
and tRPC).  The system also integrates third-party APIs for geolocation and notifications.

\subsection{11.1 Overall Architecture}

All communication between the mobile client and the backend follows a request–response
model over HTTPS, ensuring confidentiality and integrity.  Data is serialized in JSON
format and transferred through REST and type-safe tRPC endpoints.  The architecture
adopts a modular service-based design to support scalability and maintainability:

\begin{itemize}
  \item \textbf{Frontend (Client):} React Native app that sends authenticated requests
  to backend endpoints via HTTPS using the tRPC client.
  \item \textbf{Backend (Server):} Express server exposing routes through the tRPC API layer.
  It validates and processes incoming requests, interacts with the PostgreSQL database
  through Drizzle ORM, and returns structured JSON responses.
  \item \textbf{External Services:} Mapping and geolocation (Google Maps / Mapbox API),
  Expo Push Notification Service for alerts, and optional OAuth verification via McMaster
  SSO (for email validation).
\end{itemize}

\subsection{11.2 Communication Flow}

A typical data exchange involves the following sequence:

\begin{enumerate}
  \item The user performs an action in the mobile application (e.g., requests a ride,
  updates schedule, or rates a driver).
  \item The frontend constructs a tRPC call encapsulating the action data and attaches
  the user’s JWT authentication token in the header.
  \item The server authenticates the request, validates input schemas, and executes the
  corresponding service logic.
  \item The backend queries or updates PostgreSQL through Drizzle ORM and returns a
  success or error payload to the client.
  \item The client updates its UI state using React Query based on the received data.
\end{enumerate}

\subsection{11.3 API Endpoints and Message Types}

Hitchly’s backend exposes the following logical API groups:

\begin{itemize}
  \item \textbf{Auth API:} Handles user registration, login, McMaster email verification,
  and token refresh.  Exchanges encrypted credentials and JWT tokens.
  \item \textbf{User API:} Manages user profiles, schedules, and preferences.
  \item \textbf{Matching API:} Accepts commute data, performs matching algorithm, and
  returns ranked ride offers.
  \item \textbf{Trip API:} Records completed trips, generates summaries, and calculates
  cost-sharing.
  \item \textbf{Rating API:} Sends and retrieves ratings and reviews.
  \item \textbf{Notification API:} Uses Expo Push Notification service to deliver trip
  confirmations, cancellations, and updates in real time.
\end{itemize}

All messages follow the JSON structure:
\begin{verbatim}
{
  "status": "success",
  "data": { ... },
  "timestamp": "2025-01-03T12:00:00Z"
}
\end{verbatim}

\subsection{11.4 Security and Reliability}

Security and reliability are ensured through:

\begin{itemize}
  \item \textbf{Transport Security:} HTTPS with TLS 1.3 and HSTS enforcement.
  \item \textbf{Authentication:} JWT Bearer tokens included in headers of all authorized
  requests.
  \item \textbf{Input Validation:} All incoming payloads validated using Zod schemas in
  tRPC to prevent injection or malformed data.
  \item \textbf{Error Handling:} Standardized error codes with human-readable messages
  returned to the client.
  \item \textbf{Rate Limiting:} Express middleware limits excessive requests to prevent
  abuse.
  \item \textbf{Retry Mechanism:} Client-side React Query retries transient network
  failures automatically.
\end{itemize}

\subsection{11.5 Design Considerations}

The communication design of Hitchly emphasizes the following principles:

\begin{itemize}
  \item \textbf{Consistency:} Shared TypeScript types across frontend and backend
  guarantee end-to-end type safety through tRPC.
  \item \textbf{Extensibility:} Additional endpoints (e.g., payment API or advanced analytics)
  can be introduced without modifying existing client modules.
  \item \textbf{Scalability:} The stateless REST/tRPC design supports horizontal scaling
  of the Express server behind a load balancer.
  \item \textbf{Low Latency:} JSON over HTTPS is lightweight; caching of common responses
  and asynchronous notification delivery minimize perceived delay.
\end{itemize}

This design ensures secure, efficient, and maintainable communication between Hitchly’s
mobile application, backend, and external services, supporting its goals of safety,
reliability, and sustainability for McMaster commuters.


\section{Timeline}
The following timeline details our implementation for Rev 0. Each task is broken down to the module level. Each tasks has roles and responsibilities defined along with estimated development duration.
\begin{table}[h!]
  \centering
  \caption{Phase 1}
  \begin{tabularx}{\textwidth}{|X|c|c|c|}
  \hline
  \textbf{Task} & \textbf{Module(s)} & \textbf{Status} & \textbf{Responsible} \\
  \hline
  
  Creating Hitchly Wireframes for UI & M2 & Done & Burhan, Sarim \\
  \hline

  Make initial version of matching algorithm & M4 & Done & Hamzah, Sarim \\
  \hline

  Set up authentication and application bones & M1 & Done & Aidan \\
  \hline
  \end{tabularx}
\end{table}

\begin{table}[h!]
  \centering
  \caption{Phase 2: Front End and User Roles}
  \begin{tabularx}{\textwidth}{|X|c|c|c|}
  \hline
  \textbf{Task} & \textbf{Module(s)} & \textbf{Status} & \textbf{Responsible} \\
  \hline

  Develop login, registration, and session flows & M1 & Weeks 14--16 & Aidan \\
  \hline

  Build user profile screens and preferences interface & M2 & Weeks 14--16 & Swesan \\
  \hline

  Create upload, settings, and account management pages & M2 & Weeks 14--16 & Burhan \\
  \hline

  Build navigation, layouts, and key frontend components & M2 & Weeks 14--16 & Hamzah \\
  \hline

  Integrate basic user-to-database connections for profile initialization & M12 & Weeks 15--17 & Sarim \\
  \hline
  \end{tabularx}
\end{table}

\begin{table}[h!]
  \centering
  \caption{Phase 3: Core Ride-Share Logic and Trip Flow}
  \begin{tabularx}{\textwidth}{|X|c|c|c|}
  \hline
  \textbf{Task} & \textbf{Module(s)} & \textbf{Status} & \textbf{Responsible} \\
  \hline

  Implement matching algorithm and rider–driver compatibility logic 
  & M4 & Weeks 18--21 & Sarim, Hamzah \\
  \hline

  Build Routing \& Trip creation (origin, destination, seat count) 
  & M3 & Weeks 18--21 & Burhan \\
  \hline

  Implement Scheduling module (recurring rides, time windows) 
  & M5 & Weeks 19--22 & Aidan \\
  \hline

  Integrate Payment \& Cost estimation (distance, time, pricing model) 
  & M9 & Weeks 20--22 & Swesan \\
  \hline

  Build Pricing module (fare calculations, dynamic pricing rules) 
  & M11 & Weeks 20--23 & Swesan \\
  \hline

  Add Rating \& Feedback system for both riders and drivers 
  & M7 & Weeks 21--23 & Sarim \\
  \hline

  Implement Admin \& Moderation tools (flagging, dispute handling) 
  & M10 & Weeks 22--24 & Burhan, Aidan \\
  \hline

  Build Notification module (push/in-app updates for trips) 
  & M6 & Weeks 21--24 & Hamzah \\
  \hline

  \end{tabularx}
\end{table}

\begin{table}[h!]
  \centering
  \caption{Phase 4: Safety, Database Integration, and Finalization}
  \begin{tabularx}{\textwidth}{|X|c|c|c|}
  \hline
  \textbf{Task} & \textbf{Module(s)} & \textbf{Status} & \textbf{Responsible} \\
  \hline

  Implement Safety \& Reporting module (incident reports, flagging) 
  & M8 & Weeks 24--26 & Burhan \\
  \hline

  Complete full Database Module integration across all features 
  & M12 & Weeks 24--27 & Aidan, Sarim \\
  \hline

  Perform end-to-end testing of all ride flows 
  & M1--M12 & Weeks 25--28 & All \\
  \hline

  Optimize performance, resolve bugs, and finalize UI polish 
  & M1--M12 & Weeks 26--29 & All \\
  \hline

  Prepare release build, documentation, and deployment setup 
  & M1--M12 & Weeks 28--30 & All \\
  \hline

  \end{tabularx}
\end{table}


    

\clearpage

\bibliographystyle {plainnat}
\bibliography{../../../refs/References}

\newpage{}

\end{document}