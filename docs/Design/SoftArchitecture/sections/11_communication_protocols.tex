\section{Design of Communication Protocols}

This section describes the design of the communication protocols implemented in Hitchly.
Communication occurs primarily between the client‐side mobile application (developed
with React Native / Expo) and the backend server (implemented with Node.js, Express,
and tRPC).  The system also integrates third-party APIs for geolocation and notifications.

\subsection{11.1 Overall Architecture}

All communication between the mobile client and the backend follows a request–response
model over HTTPS, ensuring confidentiality and integrity.  Data is serialized in JSON
format and transferred through REST and type-safe tRPC endpoints.  The architecture
adopts a modular service-based design to support scalability and maintainability:

\begin{itemize}
  \item \textbf{Frontend (Client):} React Native app that sends authenticated requests
  to backend endpoints via HTTPS using the tRPC client.
  \item \textbf{Backend (Server):} Express server exposing routes through the tRPC API layer.
  It validates and processes incoming requests, interacts with the PostgreSQL database
  through Drizzle ORM, and returns structured JSON responses.
  \item \textbf{External Services:} Mapping and geolocation (Google Maps / Mapbox API),
  Expo Push Notification Service for alerts, and optional OAuth verification via McMaster
  SSO (for email validation).
\end{itemize}

\subsection{11.2 Communication Flow}

A typical data exchange involves the following sequence:

\begin{enumerate}
  \item The user performs an action in the mobile application (e.g., requests a ride,
  updates schedule, or rates a driver).
  \item The frontend constructs a tRPC call encapsulating the action data and attaches
  the user’s JWT authentication token in the header.
  \item The server authenticates the request, validates input schemas, and executes the
  corresponding service logic.
  \item The backend queries or updates PostgreSQL through Drizzle ORM and returns a
  success or error payload to the client.
  \item The client updates its UI state using React Query based on the received data.
\end{enumerate}

\subsection{11.3 API Endpoints and Message Types}

Hitchly’s backend exposes the following logical API groups:

\begin{itemize}
  \item \textbf{Auth API:} Handles user registration, login, McMaster email verification,
  and token refresh.  Exchanges encrypted credentials and JWT tokens.
  \item \textbf{User API:} Manages user profiles, schedules, and preferences.
  \item \textbf{Matching API:} Accepts commute data, performs matching algorithm, and
  returns ranked ride offers.
  \item \textbf{Trip API:} Records completed trips, generates summaries, and calculates
  cost-sharing.
  \item \textbf{Rating API:} Sends and retrieves ratings and reviews.
  \item \textbf{Notification API:} Uses Expo Push Notification service to deliver trip
  confirmations, cancellations, and updates in real time.
\end{itemize}

All messages follow the JSON structure:
\begin{verbatim}
{
  "status": "success",
  "data": { ... },
  "timestamp": "2025-01-03T12:00:00Z"
}
\end{verbatim}

\subsection{11.4 Security and Reliability}

Security and reliability are ensured through:

\begin{itemize}
  \item \textbf{Transport Security:} HTTPS with TLS 1.3 and HSTS enforcement.
  \item \textbf{Authentication:} JWT Bearer tokens included in headers of all authorized
  requests.
  \item \textbf{Input Validation:} All incoming payloads validated using Zod schemas in
  tRPC to prevent injection or malformed data.
  \item \textbf{Error Handling:} Standardized error codes with human-readable messages
  returned to the client.
  \item \textbf{Rate Limiting:} Express middleware limits excessive requests to prevent
  abuse.
  \item \textbf{Retry Mechanism:} Client-side React Query retries transient network
  failures automatically.
\end{itemize}

\subsection{11.5 Design Considerations}

The communication design of Hitchly emphasizes the following principles:

\begin{itemize}
  \item \textbf{Consistency:} Shared TypeScript types across frontend and backend
  guarantee end-to-end type safety through tRPC.
  \item \textbf{Extensibility:} Additional endpoints (e.g., payment API or advanced analytics)
  can be introduced without modifying existing client modules.
  \item \textbf{Scalability:} The stateless REST/tRPC design supports horizontal scaling
  of the Express server behind a load balancer.
  \item \textbf{Low Latency:} JSON over HTTPS is lightweight; caching of common responses
  and asynchronous notification delivery minimize perceived delay.
\end{itemize}

This design ensures secure, efficient, and maintainable communication between Hitchly’s
mobile application, backend, and external services, supporting its goals of safety,
reliability, and sustainability for McMaster commuters.