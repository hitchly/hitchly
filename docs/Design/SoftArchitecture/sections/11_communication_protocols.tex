\section{Design of Communication Protocols}

This section describes the design of the communication protocols implemented in Hitchly.
Communication occurs primarily between the client-side mobile application and the backend 
server infrastructure. The system also integrates third-party interfaces for geolocation 
and external notification services.

\subsection{11.1 Overall Architecture}

All communication between the mobile client and the backend follows a request–response
model over secure transport channels, ensuring confidentiality and integrity. Data is 
serialized into a structured format and transferred through defined API endpoints. 
The architecture adopts a modular service-based design to support scalability and 
maintainability:

\begin{itemize}
  \item \textbf{Frontend (Client):} Mobile application interface that sends authenticated 
  requests to backend endpoints via secure transport protocols.
  \item \textbf{Backend (Server):} Server-side application exposing routes through an 
  API layer. It validates and processes incoming requests, interacts with the 
  persistence layer, and returns structured responses.
  \item \textbf{External Services:} Interfaces for mapping and geolocation data, 
  push notification services for alerts, and identity verification providers 
  (e.g., institutional SSO for email validation).
\end{itemize}

\subsection{11.2 Communication Flow}

A typical data exchange involves the following sequence:

\begin{enumerate}
  \item The user initiates an action in the mobile application (e.g., requests a ride,
  updates schedule, or rates a driver).
  \item The frontend constructs a remote procedure call encapsulating the action data 
  and attaches the user’s authentication credentials in the header.
  \item The server authenticates the request, validates input schemas against defined 
  contracts, and executes the corresponding service logic.
  \item The backend queries or updates the persistence layer and returns a success 
  or error payload to the client.
  \item The client updates its local state and user interface based on the received data.
\end{enumerate}

\subsection{11.3 API Endpoints and Message Types}

Hitchly’s backend exposes the following logical API groups:

\begin{itemize}
  \item \textbf{Auth API:} Handles user registration, login, institutional email verification,
  and session management. Exchanges encrypted credentials and session tokens.
  \item \textbf{User API:} Manages user profiles, schedules, and preferences.
  \item \textbf{Matching API:} Accepts commute parameters, executes the matching algorithm, 
  and returns ranked ride offers.
  \item \textbf{Trip API:} Records completed trips, generates summaries, and calculates
  cost-sharing.
  \item \textbf{Rating API:} Submits and retrieves reputation scores and text reviews.
  \item \textbf{Notification API:} Interfaces with external push services to deliver trip
  confirmations, cancellations, and real-time updates.
\end{itemize}

All messages follow a standardized structure containing the operation status, payload data, 
and metadata:
\begin{verbatim}
{
  "status": "success",
  "data": { ... },
  "timestamp": "YYYY-MM-DDTHH:MM:SSZ"
}
\end{verbatim}

\subsection{11.4 Security and Reliability}

Security and reliability are ensured through:

\begin{itemize}
  \item \textbf{Transport Security:} Encryption in transit using TLS standards.
  \item \textbf{Authentication:} Bearer tokens or equivalent session credentials included 
  in headers of all authorized requests.
  \item \textbf{Input Validation:} All incoming payloads are validated against strict 
  schemas to prevent injection attacks or malformed data.
  \item \textbf{Error Handling:} Standardized error codes with human-readable messages
  returned to the client.
  \item \textbf{Rate Limiting:} Middleware limits excessive requests to prevent abuse 
  and ensure service availability.
  \item \textbf{Retry Mechanism:} Client-side logic automatically retries transient 
  network failures to improve resilience.
\end{itemize}

\subsection{11.5 Design Considerations}

The communication design of Hitchly emphasizes the following principles:

\begin{itemize}
  \item \textbf{Consistency:} Shared type definitions across frontend and backend 
  guarantee end-to-end type safety and contract adherence.
  \item \textbf{Extensibility:} Additional endpoints (e.g., payment interfaces or advanced 
  analytics) can be introduced without modifying existing client modules.
  \item \textbf{Scalability:} The stateless architectural design supports horizontal scaling 
  of the application server.
  \item \textbf{Low Latency:} Lightweight message serialization and asynchronous 
  notification delivery minimize perceived user delay.
\end{itemize}

This design ensures secure, efficient, and maintainable communication between Hitchly’s
mobile application, backend, and external services, supporting its goals of safety,
reliability, and sustainability for commuters.