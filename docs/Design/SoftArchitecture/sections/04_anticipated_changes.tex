\section{Anticipated and Unlikely Changes} \label{SecChange}

This section lists possible changes to the system. According to the likeliness
of the change, the possible changes are classified into two
categories. Anticipated changes are listed in Section \ref{SecAchange}, and
unlikely changes are listed in Section \ref{SecUchange}.

\subsection{Anticipated Changes} \label{SecAchange}

Anticipated changes are the source of the information that is to be hidden
inside the modules. Ideally, changing one of the anticipated changes will only
require changing the one module that hides the associated decision. The approach
adapted here is called design for
change.

\begin{description}
\item[\refstepcounter{acnum} \actheacnum \label{acMatchingAlgorithm}:] \textbf{Matching Algorithm:} The matching algorithm focuses on 
the goal of finding the best available drivers for the riders. This algorithm takes into account the user's location, timetable, time, and distance. For instance, it matches a rider with a driver that has the same timetable and lives in the same area. The current algorithm will be simply based on these four parameters, and it will have to change in the future to consider more parameters to increase its efficiency and accuracy. Moreover, subtle changes to the algorithm may also be expected for further optimization of its complexity and speed. 
\item[\refstepcounter{acnum} \actheacnum \label{acUserInterface}:] \textbf{User Interface:} As the application proceeds to go through its initial phase of development and use, usability testing is expected to be done to ensure user satisfaction. As the users provide feedback, changes will be made to the user interface in the future phases of development. Moreover, even after that, changes will be made based on accessibility requirements in the future. The layout and navigation are the two main things from the user interaction that would need the most changes as per user's feedback.   
\item[\refstepcounter{acnum} \actheacnum \label{acDB}:] \textbf{Database Schema:} As the app updates with new features, the current method of analytics may change. New parameters would be considered for the matching algorithm, and new features may be added based on future user needs. Therefore, changes may need to be made to the database schemas
\item[\refstepcounter{acnum} \actheacnum \label{acPayment}:] \textbf{Payment Processing:} Additional payment processors and pricing formulas will be used to determine the price estimate for drivers. An addition of parameters to estimate the cost would cause the algorithms to change.  
\item[\refstepcounter{acnum} \actheacnum \label{acDevEnvioremnt}:] \textbf{Development Environment:} Changes could be made to the CI/CD configuration to improve the reliability and stability of the system.
\end{description}


\subsection{Unlikely Changes} \label{SecUchange}

The module design should be as general as possible. However, a general system is
more complex. Sometimes this complexity is not necessary. Fixing some design
decisions at the system architecture stage can simplify the software design. If
these decision should later need to be changed, then many parts of the design
will potentially need to be modified. Hence, it is not intended that these
decisions will be changed.

\begin{description}
\item[\refstepcounter{ucnum} \uctheucnum \label{ucDevices}:] \textbf{Devices:} Hitchly is an application that is developed for mobile devices only. A version that is compatible to run on a desktop is an unlikely change as it requires major changes to the software design. 
\item[\refstepcounter{ucnum} \uctheucnum \label{ucLanguage}:] \textbf{Language Support:} The application is primarily based on the English language. Adding additional languages to the interface would require a major design decision.
\item[\refstepcounter{ucnum} \uctheucnum \label{ucProgramming}:] \textbf{Programming Language and Technologies:} The core of this application is supposed to be developed using React Native, Typescript, and SQL. Making changes to how the application is developed is an unlikely change as it requires a change in the entire design of the application.
\item[\refstepcounter{ucnum} \uctheucnum \label{ucVerification}:] \textbf{Verification Requirement:} This app is strictly restricted for McMaster affiliated users as the entire concept of this application is designed for them. It is not subject to removal. Verification of the users will be their McMaster affiliated email.
\end{description}