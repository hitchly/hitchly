\section{Database Module}

\subsection{Module}
This module is responsible for handling all database-related functionalities and ensuring smooth communication between the database servers for retrieving and updating data. This includes user information, ride details, transaction logs, etc. 

\paragraph{Frontend}
N/A

\paragraph{Backend}
The backend layer of this module handles CRUD operations for core system entities and manages communication between the application and the database for smooth retrieval and update of application data. 

\paragraph{Data}
This layer will store a schema of tables used across multiple modules. This includes: 
\begin{itemize}
    \item \textbf{User Table:} This stores all critical profile information of the user.
    \item \textbf{Ride Table:} This stores all critical ride details.
    \item \textbf{Payment Table:} This stores the user transactions for each ride.
    \item \textbf{Safety Report Table:} This stores all user safety reports.
    \item \textbf{Admin Table:} This stores the user's role details.
\end{itemize}

\subsection{Uses}
\begin{itemize}
    \item \textbf{User Profile Module:} This is used to retrieve and store user profile details.
    \item \textbf{Route \& Trip Module:} This is used to retrieve and store trip details like cost, ID, etc.
    \item \textbf{Payment Module:} This is used to retrieve and store user's transaction details.
    \item \textbf{Safety and Reporting Module:} This is used to retrieve and store user reports.
    \item \textbf{Admin Module:} This is used to retrieve and store admin details.
\end{itemize}


\subsection{Syntax}

\paragraph{Exported Constants}
\begin{itemize}
    \item \texttt{max\_connect  = 30}  
    This is the maximum number of database connections at a time. 
    \item \texttt{max\_attempt = 4}  
    Maximum attempts made to connect to db before the connection fails. 
    \item \texttt{max\_DB\_session = 30}  
    30 minutes limit before the session times out.
\end{itemize}

\paragraph{Exported Access Programs}
\par\vspace{0.5em}

\begin{center}
\resizebox{\textwidth}{!}{
\begin{tabular}{l l l l}
\toprule
\textbf{Name} & \textbf{In} & \textbf{Out} & \textbf{Exceptions} \\
\midrule
\texttt{addRecord()} & tablename, data & Boolean & NotFoundError, InvalidDataFormatException, DBTimeoutException \\
\texttt{updateRecord()} & tablename, data, record\_id & Boolean & NotFoundError, InvalidDataFormatException, DBTimeoutException \\
\texttt{deleteRecord()} & tablename, record\_id & Boolean & NotFoundError, InvalidDataFormatException, DBTimeoutException \\
\texttt{getRecord()} & tablename, data\_condition & List of data records & NotFoundError, InvalidDataFormatException, DBTimeoutException \\
\texttt{getAllRecords()} & tablename & List of data records & NotFoundError, InvalidDataFormatException, DBTimeoutException \\
\bottomrule
\end{tabular}
}
\end{center}

\subsection{Semantics}

\paragraph{State Variables}
\begin{itemize}
    \item \texttt{DBConnections: List<connections>} --- List of active database connections.
\end{itemize}

\paragraph{Environment Variables}
\begin{itemize}
    \item Secure HTTPS connection to API.
    \item Secure API connection and communication the database server.
    \item Secure connection to local/cloud backup services for data backup.    
\end{itemize}

\paragraph{Assumptions}
\begin{itemize}
    \item The Postgress database is avaiable and securely stores data.
    \item The database has sufficient storage capacity.
    \item The database automatically runs its data backup services to ensure that no data is lost.
\end{itemize}

\paragraph{Access Routine Semantics}

\texttt{addRecord()}:
\begin{itemize}
    \item \textbf{transition:} A new entry is added to the specified table.
    \item \textbf{output:} Bool (True or False, depending on the state of the transition).
    \item \textbf{exception:} NotFoundError (table not found), InvalidDataFormatException, DBTimeoutException.
\end{itemize}

\texttt{updateRecord()}:
\begin{itemize}
    \item \textbf{transition:} An existing entry is updated in the specified table.
    \item \textbf{output:} Bool (True or False, depending on the state of the transition).
    \item \textbf{exception:} NotFoundError (table or entry not found), InvalidDataFormatException, DBTimeoutException.
\end{itemize}

\texttt{deleteRecord()}:
\begin{itemize}
    \item \textbf{transition:} An existing entry is deleted in the specified table.
    \item \textbf{output:} Bool (True or False, depending on the state of the transition).
    \item \textbf{exception:} NotFoundError (table or entry not found), InvalidDataFormatException, DBTimeoutException.
\end{itemize}

\texttt{getRecord()}:
\begin{itemize}
    \item \textbf{transition:} None.
    \item \textbf{output:} Returns a list of indicated records.
    \item \textbf{exception:} NotFoundError (table not found), DBTimeoutException.
\end{itemize}

\texttt{getAllRecords()}:
\begin{itemize}
    \item \textbf{transition:} None.
    \item \textbf{output:} Returns a list of all records.
    \item \textbf{exception:} NotFoundError (table not found), DBTimeoutException.
\end{itemize}

\paragraph{Local Functions}
\begin{itemize}
    \item \texttt{db\_connect(): bool} --- This is to initialize a connection with the database service.
    \item \texttt{checkFormat(): bool} --- This function checks if the inserted data aligns with the specified schema.
\end{itemize}

% -------