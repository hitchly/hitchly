\section{Route \& Trip Module}

\subsection{Module}
The Route \& Trip Module manages trip lifecycle events, storage, and retrieval for the system. 
It enables users to publish, query, and modify trip listings that include origin, destination, departure time, and capacity. 
This module provides the authoritative source of spatial and temporal data used by the Matching Module for compatibility analysis.

\paragraph{Frontend}
Implemented within the client application, the frontend presents interactive interfaces for trip management. 
Key features include:
\begin{itemize}
    \item Input forms for trip definition (origin, destination, date, time).
    \item Visual representation of routes using an external mapping service.
    \item List and detail views for scheduled, active, and completed trips.
    \item Interface elements for seat management and cancellation.
    \item Input validation for mandatory fields and logical time constraints.
    \item Communication with the backend via secure data bindings.
\end{itemize}

\paragraph{Backend}
The backend implements business logic and trip operations through defined service endpoints. 
It is responsible for managing trip state transitions and enforcing data integrity. 
Primary responsibilities include:
\begin{itemize}
    \item Exposing interface methods for trip creation, retrieval, and deletion.
    \item Validating parameters (geographic bounds, capacity limits, temporal logic).
    \item Associating trips with authenticated user identities.
    \item Enforcing role-based access control (e.g., driver privileges).
    \item Broadcasting state changes to dependent modules (Matching, Scheduling).
\end{itemize}

\paragraph{Data}
Trip data is persisted in the relational database management system. 
The schema supports referential integrity between users, trips, and derived entities. 
Primary entities include:
\begin{itemize}
    \item \texttt{TripEntity} --- stores metadata: origin, destination, timestamps, capacity, and driver reference.
    \item \texttt{RequestEntity} --- records ride intent from passengers.
    \item \texttt{RouteEntity} --- caches geometry and distance metrics for optimization.
\end{itemize}
Each record maintains audit timestamps, and all modifications are validated against foreign key constraints referencing the user entity.

\subsection{Uses}
This module is used by the system to manage the core domain object: the trip. 
It interacts directly with:
\begin{itemize}
    \item \textbf{User Profile Module} - links trips to driver and rider profiles.
    \item \textbf{Matching Module} - provides trip data for compatibility algorithms.
    \item \textbf{Scheduling Module} - integrates recurring patterns into discrete trip entries.
    \item \textbf{Notification Module} - triggers alerts for status changes.
\end{itemize}
It serves as the foundation for dynamic route pairing and coordination within the system.

\subsection{Syntax}

\paragraph{Exported Constants}
\begin{itemize}
    \item \texttt{MAX\_SEATS = 5} - Maximum allowable passenger capacity per vehicle.
    \item \texttt{TIME\_WINDOW\_MIN = 15} - Minimum lead time (in minutes) for trip creation.
\end{itemize}

\paragraph{Exported Access Programs}
\par\vspace{0.5em}

\begin{center}
\resizebox{\textwidth}{!}{
\begin{tabular}{l l l l}
\toprule
\textbf{Name} & \textbf{In} & \textbf{Out} & \textbf{Exceptions} \\
\midrule
\texttt{createTrip} & data: TripData & TripRecord & ValidationFailed \\
\texttt{getTrips} & userId: $\mathbb{N}$ & TripList & NotFound \\
\texttt{cancelTrip} & tripId: $\mathbb{N}$ & Confirmation & Unauthorized \\
\texttt{updateTrip} & tripId: $\mathbb{N}$, data: UpdatedFields & UpdatedTripRecord & ValidationFailed \\
\texttt{getTripById} & tripId: $\mathbb{N}$ & TripRecord & NotFound \\
\bottomrule
\end{tabular}
}
\end{center}

\subsection{Semantics}

\paragraph{State Variables}
\begin{itemize}
    \item \texttt{Trips}: Map of ($\mathbb{N} \rightarrow$ TripRecord) --- The system-wide collection of all trip records indexed by tripId.
    \item \texttt{Requests}: Map of ($\mathbb{N} \rightarrow$ RequestRecord) --- The collection of pending ride requests.
\end{itemize}

\paragraph{Environment Variables}
\begin{itemize}
    \item Geolocation hardware (GPS) for spatial validation.
    \item Network interface for data synchronization.
    \item Persistence layer for reliable storage.
\end{itemize}

\paragraph{Assumptions}
\begin{itemize}
    \item The user is authenticated and possesses necessary role privileges (e.g., Driver) before creating a trip.
    \item The external mapping service is available for route validation.
    \item The persistence layer is initialized and reachable.
\end{itemize}

\paragraph{Access Routine Semantics}

\noindent \texttt{createTrip(data)}:
\begin{itemize}
    \item \textbf{transition:} Generates a unique $id$. $Trips[id] := data$ (with status = \texttt{active}).
    \item \textbf{output:} Returns $Trips[id]$.
    \item \textbf{exception:} \texttt{ValidationFailed} if $data$ contains invalid coordinates or past timestamps.
\end{itemize}

\noindent \texttt{getTrips(userId)}:
\begin{itemize}
    \item \textbf{transition:} None.
    \item \textbf{output:} Returns sequence $\{ t \in Trips \mid t.driverId = userId \lor t.riderId = userId \}$.
    \item \textbf{exception:} \texttt{NotFound} if result sequence is empty.
\end{itemize}

\noindent \texttt{updateTrip(tripId, data)}:
\begin{itemize}
    \item \textbf{transition:} $Trips[tripId] := Trips[tripId] \cup data$ (fields in $data$ overwrite existing).
    \item \textbf{output:} Returns updated $Trips[tripId]$.
    \item \textbf{exception:} \texttt{ValidationFailed} if $data$ violates constraints. \texttt{NotFound} if $tripId \notin Trips$.
\end{itemize}

\noindent \texttt{cancelTrip(tripId)}:
\begin{itemize}
    \item \textbf{transition:} $Trips[tripId].status := \texttt{canceled}$. Notifies dependent modules.
    \item \textbf{output:} \texttt{true}.
    \item \textbf{exception:} \texttt{Unauthorized} if the caller is not the owner of $Trips[tripId]$.
\end{itemize}

\noindent \texttt{getTripById(tripId)}:
\begin{itemize}
    \item \textbf{transition:} None.
    \item \textbf{output:} Returns $Trips[tripId]$.
    \item \textbf{exception:} \texttt{NotFound} if $tripId \notin Trips$.
\end{itemize}

\paragraph{Local Functions}
\begin{itemize}
    \item \texttt{calculateRouteDistance(origin, destination): $\mathbb{R}$} --- Computes geodesic or routed distance between points.
    \item \texttt{validateTripInput(data: TripData): boolean} --- Checks logical consistency of input fields.
    \item \texttt{filterTripsByTime(trips: TripList, window: Interval): TripList} --- Filters sequence by temporal bounds.
\end{itemize}