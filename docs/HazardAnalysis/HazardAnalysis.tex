\documentclass{article}

\usepackage{booktabs}
\usepackage{tabularx}
\usepackage{hyperref}

\hypersetup{
    colorlinks=true,       % false: boxed links; true: colored links
    linkcolor=red,          % color of internal links (change box color with linkbordercolor)
    citecolor=green,        % color of links to bibliography
    filecolor=magenta,      % color of file links
    urlcolor=cyan           % color of external links
}

\title{Hazard Analysis\\\progname}

\author{\authname}

\date{}

\input{../Comments}
%% Common Parts

\newcommand{\progname}{Software Engineering} % PUT YOUR PROGRAM NAME HERE
\newcommand{\authname}{Team \#16, The Chill Guys
\\ Hamzah Rawasia
\\ Sarim Zia
\\ Aidan Froggatt
\\ Swesan Pathmanathan
\\ Burhanuddin Kharodawala} % AUTHOR NAMES                  

\usepackage{hyperref}
    \hypersetup{colorlinks=true, linkcolor=blue, citecolor=blue, filecolor=blue,
                urlcolor=blue, unicode=false}
    \urlstyle{same}
                                


\begin{document}

\maketitle
\thispagestyle{empty}

~\newpage

\pagenumbering{roman}

\begin{table}[hp]
\caption{Revision History} \label{TblRevisionHistory}
\begin{tabularx}{\textwidth}{llX}
\toprule
\textbf{Date} & \textbf{Developer(s)} & \textbf{Change}\\
\midrule
Date1 & Name(s) & Description of changes\\
Date2 & Name(s) & Description of changes\\
... & ... & ...\\
\bottomrule
\end{tabularx}
\end{table}

~\newpage

\tableofcontents

~\newpage

\pagenumbering{arabic}

\wss{You are free to modify this template.}

\section{Introduction}

\wss{You can include your definition of what a hazard is here.}

\section{Scope and Purpose of Hazard Analysis}

\wss{You should say what \textbf{loss} could be incurred because of the
hazards.}

\section{System Boundaries and Components}

\wss{Dividing the system into components will help you brainstorm the hazards.
You shouldn't do a full design of the components, just get a feel for the major
ones.  For projects that involve hardware, the components will typically include
each individual piece of hardware.  If your software will have a database, or an
important library, these are also potential components.}

\section{Critical Assumptions}

\wss{These assumptions that are made about the software or system.  You should
minimize the number of assumptions that remove potential hazards.  For instance,
you could assume a part will never fail, but it is generally better to include
this potential failure mode.}

\section{Failure Mode and Effect Analysis}

\wss{Include your FMEA table here. This is the most important part of this document.}
\wss{The safety requirements in the table do not have to have the prefix SR.
The most important thing is to show traceability to your SRS. You might trace to
requirements you have already written, or you might need to add new
requirements.}
\wss{If no safety requirement can be devised, other mitigation strategies can be
entered in the table, including strategies involving providing additional
documentation, and/or test cases.}

\section{Safety and Security Requirements}

\wss{Newly discovered requirements.  These should also be added to the SRS.  (A
rationale design process how and why to fake it.)}

\section{Roadmap}

\wss{Which safety requirements will be implemented as part of the capstone timeline?
Which requirements will be implemented in the future?}

\newpage{}

\section*{Appendix --- Reflection}

\wss{Not required for CAS 741}

\input{../Reflection.tex}

\begin{enumerate}
    \item What went well while writing this deliverable? \\
    This document was very helpful as it provided a template to identify potential risks associated with the system and the users. The brainstorming session allowed for a diverse set of ideas to be shared. This was very helpful in setting the basis for this document. We were able to not only come up with hazards that directly affect the users, but we also identified hazards related to the system that would indirectly affect the users, too. This helped us understand the importance of system failures and their effects on users.
    
    \item What pain points did you experience during this deliverable, and how
    did you resolve them?\\
    Initially, we weren't able to decide on how deeply we wanted to discuss each hazard. We had discussed a lot of hazards initially, but it became difficult to decide which ones to keep and which ones to remove. We came up with the strategy to prioritize the hazards and then check for similarities to avoid redundancy. This helped us ensure that all of the critical hazards were covered and that there were no redundancies.
    
    \item Which of your listed risks had your team thought of before this
    deliverable, and which did you think of while doing this deliverable? For
    the latter ones (ones you thought of while doing the Hazard Analysis), how
    did they come about?\\
    One of the obvious ones that arises with platforms like these includes user safety and security. These ideas came from the past experiences of the teammates. The risk of having an unverified driver, breach of data, incorrect ride matches, and fake ratings and reviews, were listed before this deliverable. The rest came up while working on this deliverable. We did some research on similar applications and brainstormed ideas focusing on Hazards that could indirectly affect the user. This deliverable set the basis for considering all of the components of the application. That is when we came up with some of the system failure hazards like notification delays, loss of trip history data, location api failure, and system unavailability.   
    
    \item Other than the risk of physical harm (some projects may not have any
    appreciable risks of this form), list at least 2 other types of risk in
    software products. Why are they important to consider?\\
    \begin{enumerate}
    \item \textbf{Risk of Data Breach:} Software platforms in general hold a lot of user data. In some cases, that can be very confidential. This harms a brand's reputation and can cause heavy penalties in some cases.  
    \item \textbf{Server Crashes and Unusual Downtime:} This is another critical hazard associated with software platforms which can leave a user stranded for hours and harm the brand's image. This can cause users to move to the platforms, and cause the platform to lose users permanently.
    \end{enumerate}



\end{enumerate}

\end{document}