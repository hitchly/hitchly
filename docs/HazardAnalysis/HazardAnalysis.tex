\documentclass{article}

\usepackage{booktabs}
\usepackage{tabularx}
\usepackage{hyperref}
\usepackage{longtable}

\hypersetup{
    colorlinks=true,       % false: boxed links; true: colored links
    linkcolor=red,          % color of internal links (change box color with linkbordercolor)
    citecolor=green,        % color of links to bibliography
    filecolor=magenta,      % color of file links
    urlcolor=cyan           % color of external links
}

\title{Hazard Analysis\\\progname}

\author{\authname}

\date{}

%% Comments

\usepackage{color}

\newif\ifcomments\commentstrue %displays comments
%\newif\ifcomments\commentsfalse %so that comments do not display

\ifcomments
\newcommand{\authornote}[3]{\textcolor{#1}{[#3 ---#2]}}
\newcommand{\todo}[1]{\textcolor{red}{[TODO: #1]}}
\else
\newcommand{\authornote}[3]{}
\newcommand{\todo}[1]{}
\fi

\newcommand{\wss}[1]{\authornote{magenta}{SS}{#1}} 
\newcommand{\plt}[1]{\authornote{cyan}{TPLT}{#1}} %For explanation of the template
\newcommand{\an}[1]{\authornote{cyan}{Author}{#1}}

%% Common Parts

\newcommand{\progname}{ProgName} % PUT YOUR PROGRAM NAME HERE
\newcommand{\authname}{Team \#, Team Name
\\ Student 1 name
\\ Student 2 name
\\ Student 3 name
\\ Student 4 name} % AUTHOR NAMES                  

\usepackage{hyperref}
    \hypersetup{colorlinks=true, linkcolor=blue, citecolor=blue, filecolor=blue,
                urlcolor=blue, unicode=false}
    \urlstyle{same}
                                


\begin{document}

\maketitle
\thispagestyle{empty}

~\newpage

\pagenumbering{roman}

\begin{table}[hp]
\caption{Revision History} \label{TblRevisionHistory}
\begin{tabularx}{\textwidth}{llX}
\toprule
\textbf{Date} & \textbf{Developer(s)} & \textbf{Change}\\
\midrule
Date1 & Name(s) & Description of changes\\
Date2 & Name(s) & Description of changes\\
... & ... & ...\\
\bottomrule
\end{tabularx}
\end{table}

~\newpage

\tableofcontents

~\newpage

\pagenumbering{arabic}

\wss{You are free to modify this template.}

\section{Introduction}
Hitchly is an application that aims to provide a safe and reliable ride share platform to McMaster students, staff and faculty members. Anything that would become an obstacle to selling this software could be considered a hazard. Large software platforms like Hitchly handle user data and interactions between different users. These areas require attention as they are the most prone to hazards.  

\section{Scope and Purpose of Hazard Analysis}
The purpose of this document is to highlight potential hazards within components of the application and mitigation strategies to prevent those critical hazards from taking place. This FMEA stratergy is used to identify hazards related to user safety and security and system failure. This ensures that there is no risk of user safety, loss of brand reputation and loss of confidential user data. 

\section{System Boundaries and Components}

The system can be divided into the following components:

\subsection{Authentication System}
This component allows users to verify themselves as McMaster students and
 to log in to the system in order to access the application features.

\subsection{Profile System}
This component allows users to input relevant personal information such 
as their schedule, GPS location, and role (Driver or Passenger).

\subsection{Matching System}
This component uses backend logic to pair users based on their schedule, location, 
and preferences.

\subsection{Cost Calculation System}
This component calculates the total trip cost for the driver and creates 
a per-rider estimate that factors in gas, parking, and other related expenses.

\subsection{Database}
This component stores user details, ride requests, matches, historical 
data, and any other relevant information required by the system.

\subsection{Ratings and Review System}
This component allows users to rate each other after completing their 
trips and to add reviews to their profiles.


\section{Critical Assumptions}

The system follows these critical assumptions about the software and system:

\subsection{The Maps API provides up-to-date navigation for all users in the surrounding GTA region.}

\subsection{The application will only target McMaster students for the initial build.}

\subsection{The expected user base of the application will be within the planned demographic (students and university faculty).}

\subsection{API rate limits will not be exceeded under the expected user load.}

\section{Failure Mode and Effect Analysis}

% \wss{Include your FMEA table here. This is the most important part of this document.}
% \wss{The safety requirements in the table do not have to have the prefix SR.
% The most important thing is to show traceability to your SRS. You might trace to
% requirements you have already written, or you might need to add new
% requirements.}
% \wss{If no safety requirement can be devised, other mitigation strategies can be
% entered in the table, including strategies involving providing additional
% documentation, and/or test cases.}

The following FMEA worksheet identifies possible hazards associated with Hitchly, their causes,
effects, detection methods, and recommended mitigation actions. Traceability to the System
Requirements Specification (SRS) is included where relevant.

\subsection{FMEA Worksheet}
\begingroup
\footnotesize
\begin{longtable}{|p{0.5cm}|p{2cm}|p{2cm}|p{2cm}|p{2cm}|p{2cm}|p{2cm}|}
\caption{Failure Modes and Effects Analysis for Hitchly} \label{tbl:FMEA} \\ 
\hline
\textbf{Ref.} & \textbf{Component} & \textbf{Failure Mode} & \textbf{Causes of Failure} & \textbf{Effects of Failure} & \textbf{Detection Controls} & \textbf{Recommended Action / SRS Ref.} \\ \hline
\endfirsthead
\hline
\textbf{Ref.} & \textbf{Component} & \textbf{Failure Mode} & \textbf{Causes of Failure} & \textbf{Effects of Failure} & \textbf{Detection Controls} & \textbf{Recommended Action / SRS Ref.} \\ \hline
\endhead
\hline
\endfoot

5.1 & Authentication \& Access Control & Unauthorized user gains access & Weak verification of McMaster email; credential theft & Privacy/safety risk, personal commute data exposed & Domain restriction; failed login logging & Enforce 2FA; lockouts after failed attempts (SRS S.3.2 Auth API) \\ \hline

5.2 & Driver Verification & Unverified driver offers rides & Failure to validate license/vehicle data & Rider safety compromised & Admin checks; incomplete profile flags & Require license approval before matching (SRS S.3.1 Driver Registration) \\ \hline

5.3 & Matching Engine & Incorrect ride matches & Algorithm error; timetable parsing bug; corrupted location input & Riders matched incorrectly; missed/unsafe rides & Backend logs; user feedback reports & Unit/integration testing; fallback to safe “no match” state (SRS S.6 Matching Engine) \\ \hline

5.4 & Notifications & Missed or delayed notifications & Push service outage; API failure & Users unaware of confirmations/cancellations & Health checks; delivery logs & Retry queue; allow in-app refresh (SRS S.3.1 Notifications) \\ \hline

5.5 & Trip Data \& History & Trip not recorded or lost & DB outage; sync failure & Loss of audit trail; disputes over costs or safety & DB integrity checks; monitoring & Regular backups; retries on failure; notify user (SRS S.3.2 Trip API) \\ \hline

5.6 & Ratings \& Reviews & Malicious/fake reviews & Collusion or spamming & Misleading trust scores; reduced credibility & Anomaly detection; user reporting & Rate limit submissions; only allow after confirmed trips (SRS S.3.2 Ratings API) \\ \hline

5.7 & Location Services & Incorrect geolocation & API failure; GPS drift; spoofing & Unsafe pickups; delays; privacy risks & Compare expected vs. actual location & Allow manual override; enforce geofence on campus (SRS S.3.2 Mapping API) \\ \hline

5.8 & Data Security & Personal data leak & Server breach; weak encryption & Severe privacy violation; reputational harm & Static analysis; penetration tests & Encrypt at rest \& in transit; GDPR/PIPEDA compliance (SRS S.6.5 Code Quality) \\ \hline

5.9 & Payment (future) & Incorrect cost calculation & Bug in cost-sharing logic & Disputes between riders and drivers & Automated tests; manual audits & Provide breakdown in summary; add dispute mechanism (SRS S.3.2 Trip Summary) \\ \hline

5.10 & System Availability & App/API downtime & Server crash; container failure; network outage & Users stranded; loss of trust & Monitoring dashboards; downtime alerts & Deploy redundancy; user-facing error messaging (SRS S.6.3 System Testing) \\ \hline


\end{longtable}
\endgroup



\subsection{Conclusion}
This FMEA highlights Hitchly’s most critical hazards in authentication, ride matching,
driver verification, and data security. Some risks are mitigated directly through
\textbf{safety requirements} in the SRS (e.g., verified drivers only), while others are addressed
through \textbf{operational safeguards} such as monitoring, backups, or user documentation.
By implementing these mitigations, Hitchly can meet its goals of safety, affordability,
and sustainability for the McMaster community.

\section{Safety and Security Requirements}

\noindent \textbf{SR-1:} The system shall enforce multi-factor authentication using McMaster email verification. \\

\noindent \textbf{SR-2:} The system shall automatically lock an account after 5 consecutive failed login attempts within 10 minutes. \\

\noindent \textbf{SR-3:} The system shall verify each driver's status by validating credentials and license before enabling them to offer rides. \\

\noindent \textbf{SR-4:} The system shall record all failed matchmaking attempts and their causes for backend review and debugging. \\

\noindent \textbf{SR-5:} The system shall retry failed notication deliveries up to three times at 30-second intervals. \\

\noindent \textbf{SR-6:} The system shall include in-app refresh mechanism to allow users to manually retrieve notifications. \\

\noindent \textbf{SR-7:} The system shall perform periodic backups of all trip and booking data at intervals of 15 minutes or less. \\

\noindent \textbf{SR-8:} The system shall only allow users to submit ratings and reviews after a confirmed ride completion. \\

\noindent \textbf{SR-9:} The system shall automatically flag abnormal rating behavior for admin review. \\

\noindent \textbf{SR-10:} The system shall compare expected route location against live GPS data to detect abnomral activities, unless manually overrided by users. \\

\noindent \textbf{SR-11:} The system shall encrypt all personal and sensitive data both at rest and in transit.\\

\noindent \textbf{SR-12:} The system shall provide a dispute resolution interface for users to contest fare calculations. \\

\noindent \textbf{SR-13:} The system shall display user-facing error messages in case of system unavailability, including retry options. \\

\noindent \textbf{SR-14:} The system shall deploy redundancy to ensure 99.9\% uptime. \\

\noindent \textbf{SR-15:} The system shall comply with GDPR and PIPEDA standards for user data protection. \\

\section{Roadmap}

The safety requirements which will be implemented as part of the capstone project include:

\begin{enumerate}
    \item \textbf{Identity Verification} – Implemented in the initial release to ensure only verified McMaster users can register and log in. This is a foundational security feature.
    \item \textbf{Ride Check-ins} – Integrated during the first testing phase to ensure user accountability and to support trip completion tracking.
    \item \textbf{Data Access Control} – Implemented in parallel with backend development to ensure sensitive data (user info, trip data) is protected and only accessible by authorized roles.
    \item \textbf{Rating System} – Developed during the user interaction stage to provide feedback mechanisms for improving safety and trust among users.
\end{enumerate}

The following requirements are planned as \textbf{future milestones} beyond the capstone project timeline:

\begin{itemize}
    \item \textbf{Audit Logging} – Will be added once the system architecture is stable, to record all security-related actions for monitoring and compliance.
    \item \textbf{Data Encryption Enhancements} – Advanced encryption for all stored and transmitted user data will be prioritized for production release.
\end{itemize}

\newpage{}

\section*{Appendix --- Reflection}

\wss{Not required for CAS 741}

The purpose of reflection questions is to give you a chance to assess your own
learning and that of your group as a whole, and to find ways to improve in the
future. Reflection is an important part of the learning process.  Reflection is
also an essential component of a successful software development process.  

Reflections are most interesting and useful when they're honest, even if the
stories they tell are imperfect. You will be marked based on your depth of
thought and analysis, and not based on the content of the reflections
themselves. Thus, for full marks we encourage you to answer openly and honestly
and to avoid simply writing ``what you think the evaluator wants to hear.''

Please answer the following questions.  Some questions can be answered on the
team level, but where appropriate, each team member should write their own
response:


\begin{enumerate}
    \item What went well while writing this deliverable? \\
    
    Burhan: \\
    This document was very helpful as it provided a template to identify potential risks associated with the system and the users. The brainstorming session allowed for a diverse set of ideas to be shared. This was very helpful in setting the basis for this document. We were able to not only come up with hazards that directly affect the users, but we also identified hazards related to the system that would indirectly affect the users, too. This helped us understand the importance of system failures and their effects on users.
\\

Sarim: \\
We were able to identify genuine potential risks that would arise in our project timeline rather than blindly writing a document that would not be of use to us. This is important for us since early identification of risks and hazards is important and ensures we are able to avoid them. 
\\
    
    \item What pain points did you experience during this deliverable, and how
    did you resolve them?\\

    Burhan: \\
    Initially, we weren't able to decide on how deeply we wanted to discuss each hazard. We had discussed a lot of hazards initially, but it became difficult to decide which ones to keep and which ones to remove. We came up with the strategy to prioritize the hazards and then check for similarities to avoid redundancy. This helped us ensure that all of the critical hazards were covered and that there were no redundancies.
\\    
Sarim: \\
Understanding some of the terminology for my assigned sections was a hurdle I faced for this deliverable. When it came to writing the critical assumptions, I struggled to understand what it was referring to. However, since I started the document early, I was able to identify this issue early, and ask our TA during our meeting.
\\    

    \item Which of your listed risks had your team thought of before this
    deliverable, and which did you think of while doing this deliverable? For
    the latter ones (ones you thought of while doing the Hazard Analysis), how
    did they come about?\\
    One of the obvious ones that arises with platforms like these includes user safety and security. These ideas came from the past experiences of the teammates. The risk of having an unverified driver, breach of data, incorrect ride matches, and fake ratings and reviews, were listed before this deliverable. The rest came up while working on this deliverable. We did some research on similar applications and brainstormed ideas focusing on Hazards that could indirectly affect the user. This deliverable set the basis for considering all of the components of the application. That is when we came up with some of the system failure hazards like notification delays, loss of trip history data, location api failure, and system unavailability.   
    
    \item Other than the risk of physical harm (some projects may not have any
    appreciable risks of this form), list at least 2 other types of risk in
    software products. Why are they important to consider?\\
    \begin{enumerate}
    \item \textbf{Risk of Data Breach:} Software platforms in general hold a lot of user data. In some cases, that can be very confidential. This harms a brand's reputation and can cause heavy penalties in some cases.  
    \item \textbf{Server Crashes and Unusual Downtime:} This is another critical hazard associated with software platforms which can leave a user stranded for hours and harm the brand's image. This can cause users to move to the platforms, and cause the platform to lose users permanently.
    \end{enumerate}



\end{enumerate}

\end{document}