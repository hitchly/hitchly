\documentclass{article}

\usepackage{booktabs}
\usepackage{tabularx}
\usepackage{hyperref}
\usepackage{longtable}

\hypersetup{
    colorlinks=true,       % false: boxed links; true: colored links
    linkcolor=red,          % color of internal links (change box color with linkbordercolor)
    citecolor=green,        % color of links to bibliography
    filecolor=magenta,      % color of file links
    urlcolor=cyan           % color of external links
}

\title{Hazard Analysis\\\progname}

\author{\authname}

\date{}

\input{../Comments}
%% Common Parts

\newcommand{\progname}{Software Engineering} % PUT YOUR PROGRAM NAME HERE
\newcommand{\authname}{Team \#16, The Chill Guys
\\ Hamzah Rawasia
\\ Sarim Zia
\\ Aidan Froggatt
\\ Swesan Pathmanathan
\\ Burhanuddin Kharodawala} % AUTHOR NAMES                  

\usepackage{hyperref}
    \hypersetup{colorlinks=true, linkcolor=blue, citecolor=blue, filecolor=blue,
                urlcolor=blue, unicode=false}
    \urlstyle{same}
                                


\begin{document}

\maketitle
\thispagestyle{empty}

~\newpage

\pagenumbering{roman}

\begin{table}[hp]
\caption{Revision History} \label{TblRevisionHistory}
\begin{tabularx}{\textwidth}{llX}
\toprule
\textbf{Date} & \textbf{Developer(s)} & \textbf{Change}\\
\midrule
Date1 & Name(s) & Description of changes\\
Date2 & Name(s) & Description of changes\\
... & ... & ...\\
\bottomrule
\end{tabularx}
\end{table}

~\newpage

\tableofcontents

~\newpage

\pagenumbering{arabic}

\wss{You are free to modify this template.}

\section{Introduction}
Hitchly is an application that aims to provide a safe and reliable ride share platform to McMaster students, staff and faculty members. Anything that would become an obstacle to selling this software could be considered a hazard. Large software platforms like Hitchly handle user data and interactions between different users. These areas require attention as they are the most prone to hazards.  

\section{Scope and Purpose of Hazard Analysis}
The purpose of this document is to highlight potential hazards within components of the application and mitigation strategies to prevent those critical hazards from taking place. This FMEA stratergy is used to identify hazards related to user safety and security and system failure. This ensures that there is no risk of user safety, loss of brand reputation and loss of confidential user data. 

\section{System Boundaries and Components}

\wss{Dividing the system into components will help you brainstorm the hazards.
You shouldn't do a full design of the components, just get a feel for the major
ones.  For projects that involve hardware, the components will typically include
each individual piece of hardware.  If your software will have a database, or an
important library, these are also potential components.}

\section{Critical Assumptions}

\wss{These assumptions that are made about the software or system.  You should
minimize the number of assumptions that remove potential hazards.  For instance,
you could assume a part will never fail, but it is generally better to include
this potential failure mode.}

\section{Failure Mode and Effect Analysis}

% \wss{Include your FMEA table here. This is the most important part of this document.}
% \wss{The safety requirements in the table do not have to have the prefix SR.
% The most important thing is to show traceability to your SRS. You might trace to
% requirements you have already written, or you might need to add new
% requirements.}
% \wss{If no safety requirement can be devised, other mitigation strategies can be
% entered in the table, including strategies involving providing additional
% documentation, and/or test cases.}

The following FMEA worksheet identifies possible hazards associated with Hitchly, their causes,
effects, detection methods, and recommended mitigation actions. Traceability to the System
Requirements Specification (SRS) is included where relevant.

\subsection{FMEA Worksheet}
\begingroup
\footnotesize
\begin{longtable}{|p{0.5cm}|p{2cm}|p{2cm}|p{2cm}|p{2cm}|p{2cm}|p{2cm}|}
\caption{Failure Modes and Effects Analysis for Hitchly} \label{tbl:FMEA} \\ 
\hline
\textbf{Ref.} & \textbf{Component} & \textbf{Failure Mode} & \textbf{Causes of Failure} & \textbf{Effects of Failure} & \textbf{Detection Controls} & \textbf{Recommended Action / SRS Ref.} \\ \hline
\endfirsthead
\hline
\textbf{Ref.} & \textbf{Component} & \textbf{Failure Mode} & \textbf{Causes of Failure} & \textbf{Effects of Failure} & \textbf{Detection Controls} & \textbf{Recommended Action / SRS Ref.} \\ \hline
\endhead
\hline
\endfoot

5.1 & Authentication \& Access Control & Unauthorized user gains access & Weak verification of McMaster email; credential theft & Privacy/safety risk, personal commute data exposed & Domain restriction; failed login logging & Enforce 2FA; lockouts after failed attempts (SRS S.3.2 Auth API) \\ \hline

5.2 & Driver Verification & Unverified driver offers rides & Failure to validate license/vehicle data & Rider safety compromised & Admin checks; incomplete profile flags & Require license approval before matching (SRS S.3.1 Driver Registration) \\ \hline

5.3 & Matching Engine & Incorrect ride matches & Algorithm error; timetable parsing bug; corrupted location input & Riders matched incorrectly; missed/unsafe rides & Backend logs; user feedback reports & Unit/integration testing; fallback to safe “no match” state (SRS S.6 Matching Engine) \\ \hline

5.4 & Notifications & Missed or delayed notifications & Push service outage; API failure & Users unaware of confirmations/cancellations & Health checks; delivery logs & Retry queue; allow in-app refresh (SRS S.3.1 Notifications) \\ \hline

5.5 & Trip Data \& History & Trip not recorded or lost & DB outage; sync failure & Loss of audit trail; disputes over costs or safety & DB integrity checks; monitoring & Regular backups; retries on failure; notify user (SRS S.3.2 Trip API) \\ \hline

5.6 & Ratings \& Reviews & Malicious/fake reviews & Collusion or spamming & Misleading trust scores; reduced credibility & Anomaly detection; user reporting & Rate limit submissions; only allow after confirmed trips (SRS S.3.2 Ratings API) \\ \hline

5.7 & Location Services & Incorrect geolocation & API failure; GPS drift; spoofing & Unsafe pickups; delays; privacy risks & Compare expected vs. actual location & Allow manual override; enforce geofence on campus (SRS S.3.2 Mapping API) \\ \hline

5.8 & Data Security & Personal data leak & Server breach; weak encryption & Severe privacy violation; reputational harm & Static analysis; penetration tests & Encrypt at rest \& in transit; GDPR/PIPEDA compliance (SRS S.6.5 Code Quality) \\ \hline

5.9 & Payment (future) & Incorrect cost calculation & Bug in cost-sharing logic & Disputes between riders and drivers & Automated tests; manual audits & Provide breakdown in summary; add dispute mechanism (SRS S.3.2 Trip Summary) \\ \hline

5.10 & System Availability & App/API downtime & Server crash; container failure; network outage & Users stranded; loss of trust & Monitoring dashboards; downtime alerts & Deploy redundancy; user-facing error messaging (SRS S.6.3 System Testing) \\ \hline


\end{longtable}
\endgroup



\subsection{Conclusion}
This FMEA highlights Hitchly’s most critical hazards in authentication, ride matching,
driver verification, and data security. Some risks are mitigated directly through
\textbf{safety requirements} in the SRS (e.g., verified drivers only), while others are addressed
through \textbf{operational safeguards} such as monitoring, backups, or user documentation.
By implementing these mitigations, Hitchly can meet its goals of safety, affordability,
and sustainability for the McMaster community.

\section{Safety and Security Requirements}

\wss{Newly discovered requirements.  These should also be added to the SRS.  (A
rationale design process how and why to fake it.)}

\section{Roadmap}

\wss{Which safety requirements will be implemented as part of the capstone timeline?
Which requirements will be implemented in the future?}

\newpage{}

\section*{Appendix --- Reflection}

\wss{Not required for CAS 741}

\input{../Reflection.tex}

\begin{enumerate}
    \item What went well while writing this deliverable? 
    \item What pain points did you experience during this deliverable, and how
    did you resolve them?
    \item Which of your listed risks had your team thought of before this
    deliverable, and which did you think of while doing this deliverable? For
    the latter ones (ones you thought of while doing the Hazard Analysis), how
    did they come about?
    \item Other than the risk of physical harm (some projects may not have any
    appreciable risks of this form), list at least 2 other types of risk in
    software products. Why are they important to consider?
\end{enumerate}

\end{document}